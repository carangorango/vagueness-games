\documentclass[a4paper]{article}
\usepackage[utf8]{inputenc}
\usepackage{bm}

\title{Meta-dynamics: a sketch}
\author{José Pedro Correia}
\date{\today}

\begin{document}
\maketitle

Say that we want to test different dynamics against each other in one population of agents.
Each dynamic is used by a certain percentage of the population.
Each sub-population is characterized by a sender (receiver) strategy which is evolved according to its dynamic, but they interact with each other, so each strategy's expected utility is calculated against the whole population.
I will here assume that senders and receivers are separate sub-populations.

We can denote the percentage of sender population using dynamic $i \in I$ by $S_i$ and the percentage of receiver population using the same dynamic by $R_i$.
Strategies are also indexed by the dynamic, so we have $\sigma_i$ and $\rho_i$ for each $i \in I$.
Dynamics are functions $D_i$ that calculate a new generation of strategies based on the existing ones (this could benefit from a more formal characterization).

The calculation of a new generation now applies to both the strategies themselves, as to the dynamics.
Sender strategy $\sigma_i$ gets updated according to its dynamic $D_i$:
$$
\sigma_i^\prime(m \mid t) = D_i(\sigma_i, EU(m \mid t))
$$

For example, if $D_3$ is the replicator dynamic, we would have the following:
$$
\sigma_3^\prime(m \mid t) = \sigma_3(m \mid t) \cdot \frac{EU(m \mid t)}{\sum_{m^\prime \in M} EU(m^\prime \mid t)}
$$

One important point is that expected utility for the sender strategy of one dynamic is calculated against the receiver strategies of all other dynamics:
$$
EU(m \mid t) = \sum_{i \in I} R_i \cdot \sum_{t^\prime \in T} \rho_i(t^\prime \mid m) \cdot U(t, t^\prime)
$$

Then we also update the percentage of senders using a certain dynamic.
In this case, using the replicator equation as meta-dynamic, we get the following:
$$
S_i^\prime = S_i \cdot \frac{EU(\sigma_i)}{\sum_{i^\prime \in I} EU(\sigma_{i^\prime})}
$$
where the expected utility of a whole sender strategy of one dynamic is also calculated against the receiver strategies of all other dynamics:
$$
EU(\sigma) = \sum_{i \in I} R_i \cdot \sum_{t, t^\prime \in T} \sum_{m \in M} P(t) \cdot \sigma(m \mid t) \cdot \rho_i(t^\prime \mid m) \cdot U(t, t^\prime)
$$

We can make similar formulations, \emph{mutatis mutandis}, for the receiver populations.
	
\end{document}
