\documentclass[a4paper]{article}

\usepackage[natbib=true,style=authoryear-comp,backend=biber,doi=false,url=false,isbn=false]{biblatex}
\bibliography{paper}

\usepackage{todonotes}
\usepackage{a4wide}
\usepackage{amsmath}
\usepackage{amssymb}
\usepackage{syllogism}

\begin{document}

\title{Natural sources of vagueness and their implications}
\author{Michael Franke \and Jos\'e Pedro Correia}
\date{}

\maketitle

\begin{abstract}
A vexing puzzle about vagueness, rationality and evolution runs, in crude abbreviation, as follows: vague language use is demonstrably suboptimal if the goal is efficient precise and cooperative information transmission; hence rational deliberation or evolutionary selection should, under this assumed goal, eradicate vagueness from language use.
Since vagueness is persistent in all human languages, something has to give.
In this paper, we investigate a number of reasons why and mechanisms how vagueness may come into the picture in formal models of rational or evolutionary optimal signaling.
We show how uncertainty about not only the linguistic practices of others, but also about the world itself can lead to vagueness, and how, given vagueness, natural linguistic practices are likely to create more reason for vagueness.
We explore the consequences of these reasons and mechanism for a notion of meaning and a notion of language.
\end{abstract}

\tableofcontents

\section{Vagueness}
\label{sec:vagueness}

The classical philosophical problem of vagueness is most starkly embodied by the sorites paradox.
The original formulation is attributed to Eubulides, an ancient Megarian philosopher~\parencite{sorensen_sorites_2009}, and uses the example of a heap of sand: if $1 000 000$ grains of sand piled up together clearly form a heap, and removing one grain of sand can never make a heap into a non-heap, then if we remove one grain, the remaining $999 999$ also clearly form a heap; by repeated applications of this reasoning we can remove almost all grains of sand one by one and are lead to acknowledge that $1$ single grain of sand also forms a heap.
This paradox is interesting because it can be made general and re-applied to many other words besides `heap'.
In the general formulation, we start with an $x$ to which a certain predicate $P$ clearly applies.
We observe that a certain transformation $f$ has no impact on whether something is or is not $P$.
Thus we are lead to acknowledge that the predicate $P$ also applies to $y = f(x)$.
Moreover, we can apply the same reasoning subsequently to $f(x)$ and repeat that as many times as the transformation is still applicable.
The paradox arises when $n$ repeated applications of the transformation $f$ lead us from $x$ to $z = f^n(x)$, where $z$ is something for which originally we would say $P$ clearly does not apply, but since we see it as difficult to reject either the premisses or the reasoning, we seem to be forced to accept that $P$ \emph{does in fact apply} to $x$.
The paradox with the heap can thus be seen as an instance of this general formulation, if $x$ is the group of $1 000 000$ grains of sand, the transformation is removing a grain, and $z$ the state of one (or even zero) grains.
It can also be reversed: it seems reasonable to accept that piling up a grain of sand to something that is not a heap will not turn it into a heap of sand, thus 
by repeated applications of that transformation one can pile up as many grains of sand as one likes and will never obtain a heap.

Predicates for which one can find a suitable instance of the general formulation of the sorites paradox are called \emph{vague}.
Paradigmatic examples besides `heap' include `tall', `red', `bald', `tadpole', and `child'~\parencite{Keefe1997}.
An important intuition regarding these predicates that enables the sorites paradox to go through is that there does not seem to be any clear boundary demarcating the cases where the predicate applies from the cases where it doesn't, but there are nevertheless examples of both cases.
Intuitively, a person with a height of 2 meters is tall, and one with a height of 1.5 meters is not, but there is no value of height in between where one would draw the line; saying, for example, that a person with a height of 1.80 meters is tall, but a person measuring 1.79 meters is not, just seems preposterous.
Neither drawing a sharp boundary seems reasonable, nor do the logical consequences of not doing so, and in there lies the truely paradoxical issue of vagueness.
How widespread is the problem?
It is easy to come up with more examples of predicates that are based on more finely grained properties, like `tall' is intuitively based on height, for which constructing a sorites paradox would be easy.
Mereological nihilists argue that instances of the sorites can be designed for any material object that can be decomposed into small enough parts.
If we subscribe to the scientific picture of matter as composed of molecules and atoms, this applies to tables and chairs, cats and mats, and any other ordinary thing~\parencite{Unger1979}.
Any heap of molecules that we have a name for is potentially at the mercy of being hypotheticall chipped away into nothingness while still keeping its name.
The problem thus seems to be a serious one.
Bertrand Russell famously argued~\parencite*{russell_vagueness_1923} that all words, including ``the words of pure logic'', are vague when used by human beings.
In face of such a powerful paradox like the sorites, something has got to give.

Vagueness is typically seen as a challenge to a classical conception of language and meaning.
We can summarize this classical picture as such: words stand in a direct or mediated correspondence to their meanings; sentences are combinations of such words bound by logical rules; knowledge of the meanings of words and of how the rules combine them allows us to know the meaning of sentences and determine whether they are true or false.
With variations, this picture could be said to underlie theories of meaning of authors life Gottlob Frege, Bertrand Russell, Ludwig Wittgenstein (his early work), Alfred Tarski, Richard Montague, and many others, especially those working in the analytic tradition.
The problem is that vague predicates seem to lack precise boundaries; if that is the case, how can words like `tall' stand in correspondence to a well defined set of people?
And if we cannot know exactly whether a certain person is `tall' or not, as with borderline cases, how are we to determine the truth value of sentences that involve statements of tallness regarding this person?

Some argue that the problems caused by vagueness are not serious enough for us to throw away the picture of meaning outlined above.
Timothy Williamson, for example, claims that we should stick with the classical picture because ``[c]lassical logic and semantics are vastly superior to the alternatives in simplicity, power, past success, and integration with theories in other domains.''~\parencite*[162]{williamson_vagueness_1992}
In the face of vagueness, Williamson defends the so-called \emph{epistemic view}: vague terms do determine precise boundaries, we just don't have the ability to know where they are; there is a precise height above which people are tall, and below which they are not, we are just ignorant of what this value is.
%%% Could talk about Frazee, J., and Beaver, D. (2010) here
The epistemic approach is curious in that, in its attempt to save truth, it seems to actually render it sterile as an explanatory notion.
If we can never ascertain truth when it comes to vague terms, how can truth be relevant for explaining how we understand or produce language?
If the concept of truth is crucially tied to meaning, but we have no access to the former, how could we ever have learned the latter?
Williamson's reason to ignore these problems is a pragmatic one: he believes it helps us, philosophers of language and theoretical linguists, to better understand natural language.
But this reason holds only insofar as we agree with this assessment of the success of the classical picture.

Criticism of the adequacy of this picture, however, comes from various quarters.
One of the strongest can be found in the later work of philosopher Ludwig Wittgenstein\footnote{To the extent that substantial views can be said to be defended by the author. Skipping over the debate (see \cite{kahane_wittgenstein_2007} for more details), we are here assuming an interpretation of Wittgenstein as a kind of pragmatist, along the lines of the readings of Hilary Putnam~\parencite*{putnam_pragmatism_1994} and Richard Rorty~\parencite*{rorty_wittgenstein_2007}.}, most importantly in the \emph{Philosophical Investigations}~\parencite*{wittgenstein_philosophical_1953}.
Undermining several assumptions behind the classical picture and the implications thereof, Wittgenstein pushes for a paradigm shift:
\begin{quote}
The more closely we examine actual language, the greater becomes the conflict between it and our requirement.
(For the crystalline purity of logic was, of course, not something I had \emph{discovered}: it was a requirement.)
The conflict becomes intolerable; the requirement is now in danger of becoming vacuous.
-- We have got on to slippery ice where there is no friction, and so, in a certain sense, the conditions are ideal; but also, just because of that, we are unable to walk.
We want to walk: so we need \emph{friction}.
Back to the rough ground!%
~\parencite[\S 107]{wittgenstein_philosophical_1953}
\end{quote}
He argues that the ideal of exactness that is tied to the notions of truth and logic dazzles and leads to misunderstandings~\parencite*[\S 100]{wittgenstein_philosophical_1953}; it creates a haze around the workings of language~\parencite*[\S 5]{wittgenstein_philosophical_1953} that keeps us from seeing what is right in front of us.
We can only escape this predicament, not by bargaining exactness out of logic, but by turning our whole inquiry around~\parencite*[\S 108]{wittgenstein_philosophical_1953}, \emph{i.e.}~by thinking of language in terms of a different paradigm.

%%% This discussion is perhaps too detailed for the scope of this paper
% Another suggested theory that attempts to retain as much as possible from classical logic by making minimal modifications to the classical picture of semantics is \emph{supervaluationism}.
% Championed by Kit Fine~\parencite*{Fine1975}, the idea is that vague predicates can be made precise, even arbitrarily.
% Setting a boundary between tall and not tall people at 1.80 meters would be an admissible%
% \footnote{Only precisifications that meet certain constraints should be considered. We refer the reader to Fine's article~\parencite*{Fine1975} for the details.}
% \emph{precisification} of the word `tall'.
% What supervaluationism defends is that, when considering the truth value of a sentence with vague terms, we should take all admissible precisifications of those terms into account: the sentence is true if true under all of them, false if false under all of them, and neither true nor false otherwise.
% The intuition is that a person with a height of 2 meters would be considered tall under all admissible precisifications, thus we could say it is true that that person is tall; for someone measuring 1.79 this would not be the case, hypothetically being true under some precisifications and false under others.
% %In defending this approach, supervaluationism either assumes or postulates the potential existence of a boundary between the sentences for which all admissible precisifications of the vague terms therein make them true, and those for which at least one precisification makes them false.
% Although supervaluationism claims to retain classical logics, it seems to implicitly require a third truth value to account for sentences for which not all precisifications are in agreement.
% Other responses to the problem of vagueness develop alternative logics that explicitly break away from the bivalent assumption that a sentence is either true or false, and introduce additional truth values.
% These range from three-valued logics~\parencite[\emph{e.g.}][]{tye_sorites_1994} to infinite-valued degree theories~\parencite[\emph{e.g.}][]{machina_truth_1976}.

The way in how the ideal of exactness leads into misunderstandings is somewhat patent in the standard approaches to dealing with vagueness that try to hang on to as much as possible of the classical picture.
Supervaluationism, many-valued logics, and degree theories all propose changes to it in order to accommodate for vague predicates.
The proposals are, however, still within the general molds of the classical picture.
Mark Sainsbury~\parencite*{sainsbury_concepts_1999} argues that, because of that, they all fail to address an important characteristic of vague predicates: \emph{higher order vagueness}.
All the aforementioned proposals end up being committed to new artificial demarcating boundaries (\emph{e.g.}~true-under-all-precisifications versus neither true nor false versus false-under-all-precisifications, true versus indefinite versus false, true to degree 1 versus true to degree 0 versus the rest).
But a vague predicate not only fails to demarcate between the cases where it clearly applies and the ones where it clearly doesn't, it also fails to establish a boundary between the cases where it clearly applies and the borderline cases, as well as between the borderline cases and the cases where it clearly doesn't apply.
Further introducing borderline borderline cases would lead into an infinite regress.
Because of their attachment to the classical picture, the standard approaches to vagueness fail to see an important lesson, namely that ``we do not know, cannot know, and do not need to know these supposed boundaries to use language correctly''~\parencite*[256]{sainsbury_concepts_1999}.
% \begin{quote}
% But to what in our actual use of language does this division correspond?
% It looks as if, as before, it should correspond to the sentences true beyond the shadow of vagueness, those in some kind of borderline position, and those false beyond the shadow.
% But [\ldots] we do not know, cannot know, and do not need to know these supposed boundaries to use language correctly.
% Hence they cannot be included in a correct description of our language.%
% ~\parencite[256]{sainsbury_concepts_1999}
% \end{quote}
By trying to cling as much as possible to the classical picture of logic and semantics, these standard approaches are ignoring a simple observation: natural language users are sensitive to the sorites paradox, \emph{i.e.}~are able to recognize the logical inconsistency but do not have a good answer to overcome it.
Even more importantly, they apparently do not need to solve the inconsistency in order to continue using natural language productively.
Nobody ever stopped using the word `tall` after being confronted with a sorites series to deconstruct it.
Why should we develop theories of meaning that are impervious to the paradox?

The reluctance to give up truth and logic as valuable notions to explain meaning is perhaps associated with the fear of what would also consequently need to be abandoned down the line.
Talking about philosophers who defend the desirability of a classical notion of truth, Richard Rorty says:
\begin{quote}
In the past, such philosophers have typically conjoined the claim that there is universal human agreement on the supreme desirability of truth with two further premises: that truth is correspondence to reality, and that reality has an intrinsic nature (that there is, in Nelson Goodman's terms, a Way the World Is).
[\ldots]
The rise of relatively democratic, relatively tolerant, societies in the last few hundred years is said to be due to the increased rationality of modern times, where `rationality' denotes the employment of an innate truth-oriented faculty.%
~\parencite*[1]{rorty_response_2000-1}
\end{quote}
Rationality, in this picture, has truth as its guiding light and logic as the means to attain it.
The fear could be that, if we drop the picture of meaning as intimately tied to truth and logic, we lose the ground on which rationality stands, and the whole edifice would collapse.
But giving up the ideal of truth and logic as relevant explanatory notions to understand natural language does not mean giving up on rationality, or any of the other notions, altogether.
To say that language does not follow strict logical rules, or that truth is not a useful notion to guide our inquiries into meaning, is not to say that language is unstructured, meaningless, or unusable; neither is it to say that we, language users, are therefore completely irrational.
Vagueness can shed light on features of our language and our rationality, but only after we step out of the classical picture and start looking at these concepts in terms of a different paradigm.

What can such a paradigm be?
The Wittgensteinian picture of language is one of multiplicity, heterogeneity, and change.
Languages are thought of as patchworks of various language-games, with new ones continuously being added and old ones falling out of use.
These language-games, in turn, are situated uses of words in certain activities.
The metaphor emphasizes a concept of meaning as strongly linked to the use that words or signs are put to.
Meaning is thus contingent to the language-games we play, and dynamic.
The notion of language-game is also used to characterize one of Wittgenstein's methodological tools.
Following the idea that ``[i]t disperses the fog if we study the phenomena of language in primitive kinds of use in which one can clearly survey the purpose and functioning of the words''~\parencite*[\S 5]{wittgenstein_philosophical_1953}, the method is to set up such a language-game as a hypothetical scenario, or thought experiment, where language is used in a certain type of activity, and then reflect on the assumptions that underlie our interpretation of this set-up, as well as on how the scenario would play through according to those assumptions.
Jos\'e Pedro Correia~\parencite*{correia_bivalent_2013} argues that the framework of signaling games, introduced for the study of meaning by David Lewis~\parencite*{lewis_convention_1969} and later naturalized by Brian Skyrms~\parencite*{skyrms_evolution_1996,skyrms_signals_2010}, permits us to do exactly that, while reaping the benefits of a mathematical formalism and the further ability to conduct computer simulations.
From this perspective, creating a signaling game model is like setting up a language-game, only using a formulation that improves perspicuity, and conducting computer simulations is like contemplating how the game would play through according to the implications of the assumptions one built into the model.

In the following, we will be looking at signaling models of vagueness in order to provide an overview of proposals of natural sources of vagueness, what that tells us about rationality, and the implications of these ideas.
But first, let us look in more detail into how Lewis-Skyrms signaling games work, and how vagueness shows up again as a different kind of problem for these models.
%With regards to vagueness and exactness, it is also important to note the following:
%\begin{quote}
%``Inexact'' is really a reproach, and ``exact'' is praise.
%And that is to say that what is inexact attains its goal less perfectly than does what is more exact.
%So it all depends on what we call ``the goal''.
%\end{quote}

%%% This is perhaps too philosophical and off-topic
% The reluctance to give up truth and logic as fundamental concepts to understand meaning is perhaps related to a fear of losing a grip on rationality, and more importantly, on the supposed necessity of these concepts to underwrite normative statements and thus condition discourse.
% Richard Rorty describes this as follows:
% \begin{quote}
% The traditional view is that there is a core self which can look at, decide among, use, and express itself by means of, [\ldots] belief and desires.
% Further, these beliefs and desires are criticizable not simply by reference to their ability to cohere with one another, but by reference to something exterior to the network within which they are strands.
% Beliefs are, on this account, criticizable because they fail to correspond to reality.
% Desires are criticizable because they fail to correspond to the essential nature of the human self -- because they are ``irrational'' or ``unnatural''.
% \end{quote}
% Since the linguistic turn, language is seen as the medium that puts the self in contact with either reality or the nature of the self.
% Rationality is, in this picture, dependent on truth as its objective and logic as the means to attain it.
% The latter are embodied in language.
% The fear is thus that if we drop the picture of meaning as intimately tied to truth and logic, we lose the ground that rationality stands on, and thus lose the ability to make value judgments on sentences from a supposedly objective standpoint.
% Full-fledged relativism thus ensues.

\section{Signaling games and another problem of vagueness}
\label{sec:signaling-and-Lipman}

Signaling games were first introduced as models of communication by David Lewis~\parencite*{lewis_convention_1969}.
He wanted to address an argument raised by W.~V.~Quine~\parencite*{quine_truth_1936} and others against the possibility of language having had started as a conventional system: if language is a convention, it had to be originally established by an agreement; in order to establish an agreement, a convention-governed system of communication would have to already have been in place; thus, although some languages could have been established by agreement if another convention was already in place, not all of them could.
To this, Lewis retorts:
\begin{quote}
I offer this rejoinder: an agreement sufficient to create a convention need not be a transaction involving language or any other conventional activity.
All it takes is an exchange of manifestations of a propensity to conform to a regularity.%
~\parencite*[87--88]{lewis_convention_1969}
\end{quote}
In order to support this claim, Lewis studies coordination problems formalized in terms of game theory.

Coordination problems are ``situations of interdependent decision by two or more agents in which coincidence of interest predominates and in which there are two or more proper coordination equilibria''~\parencite*[24]{lewis_convention_1969}.
In game theory terms, the agents interested in the coordination are the players, the game involves each player making an independent choice from his set of available actions, a payoff is what is attributed to each player based on the choices of both.
The setup of the game, \emph{i.e.}~the available actions and the relative interests of the players, can be represented in a payoff matrix, where rows show one player's available actions, columns the other player's available actions, and each cell gives the payoff for each of the players based on its row and column combination.
For example, consider the following matrix:
\begin{center}
\begin{tabular}{r|c|c|}
\multicolumn{1}{r}{}
 & \multicolumn{1}{c}{$b_1$}
 & \multicolumn{1}{c}{$b_2$} \\ \cline{2-3}
   $a_1$ & $1,1$ & $0,0$ \\ \cline{2-3}
   $a_2$ & $0,0$ & $1,1$ \\ \cline{2-3}
\end{tabular}
\end{center}
We can see this as representing the following game: one player has actions $A = \lbrace a_1, a_2 \rbrace$ available, the other $B = \lbrace b_1, b_2 \rbrace$; they prefer to coordinate $a_1$ with $b_1$ or $a_2$ with $b_2$, thus if this is achieved each gets a payoff of $1$, otherwise they each get $0$.
Formally, all we need to define in order to characterize such a situation are the sets of actions $A$ and $B$, and a utility function $U : A \times B \rightarrow \mathbb{R}^2$ that specifies the payoff for each player given the choices of actions.

The above example is a very simple case, but it serves to illustrate Lewis' notion of convention.
The two pairs of choices $(a_1,b_1)$ and $(a_2,b_2)$ are stable coordination equilibria, because in such a scenario no player has an incentive to unilaterally change his choice: if the first player is going to choose action $a_1$, the second player would get a payoff of $0$ for switching to $b_2$, instead of $1$ from sticking to $b_1$; the same reasoning applies, mutatis mutandis, to the other player and the other equilibrium.
However, neither player has a preference between those two pairs.
Furthermore, if the same coordination problem arises again, Lewis says, we can expect precedence to induce a kind of regularity: if the players manage to coordinate on one of the two equilibria, they should be expected to repeat the choices of actions that lead to that success, thus remaining in the same equilibrium.
Not only can we expect the players in such a repeated interaction to conform to a regularity, but also to expect others to do the same.
Their preference is to remain in a certain equilibrium given that others do too.
This summarizes Lewis's notion of a convention.

In order to extend this notion to \emph{linguistic} conventions, Lewis considers situations where the actions available involve sending and receiving signals or messages.
Thus, we could think of two players with different roles.
The first player, which Lewis calls the communicator~\parencite*[130]{lewis_convention_1969} and we will here call the sender, has knowledge about which of a number of possible states of affairs obtains, and depending on this information chooses a signal to send.
The second player, which Lewis calls the audience~\parencite*[130]{lewis_convention_1969} and we will here call the receiver, has knowledge about which signal the sender chose, and based on this information chooses one of several possible responses.
A preference relation exists between responses and states of affairs, and a payoff is attributed to each player based on the choices of both.
Note that Lewis assumes that no player has any preference regarding the particular signal that is used, provided that it enables coordination.
Formally, in order to describe the setup all we need is to specify a set of possible states of affairs $T$, a set of available signals or messages $M$, a set of responses or actions $A$, and the utility function $U : T \times A \rightarrow \mathbb{R}^2$.

Despite the added dimension of the signal exchange, these so-called signaling problems can be seen as particular cases of coordination problems if we consider the players' choices to be of contingency plans or strategies.
A communicator's contingency plan, or sender strategy as we will call it, is a specification of a choice of message for each possible state of affairs.
It thus describes the sender's behavior conditional on the state of affairs that obtains.
An audience's contingency plan, or receiver strategy as we will call it, analogously specifies a choice of action for each possible message.
Thus, formally, what the sender chooses is a function $\sigma : T \rightarrow M$ and the receiver a function $\rho : M \rightarrow A$.
The expected utility $EU$ of a pair of strategies $(\sigma,\rho)$ can be calculated using the utility function between states of affairs and actions as a sum of the payoffs for all cases, \emph{i.e.}:
$$
EU(\sigma, \rho) = \sum_{t \in T} U(t, \rho(\sigma(t)))
$$
As an example, consider a game with $T = \lbrace t_1, t_2 \rbrace$, $M = \lbrace m_1, m_2 \rbrace$, $A = \lbrace a_1, a_2 \rbrace$, and the following utility matrix:
\begin{center}
\begin{tabular}{r|c|c|}
\multicolumn{1}{r}{}
 & \multicolumn{1}{c}{$a_1$}
 & \multicolumn{1}{c}{$a_2$} \\ \cline{2-3}
   $t_1$ & $1,1$ & $0,0$ \\ \cline{2-3}
   $t_2$ & $0,0$ & $1,1$ \\ \cline{2-3}
\end{tabular}
\end{center}
Possible sender and receiver strategies could be, for example, $\sigma = \lbrace t_1 \mapsto m_2, t_2 \mapsto m_1 \rbrace$ and $\rho = \lbrace m_1 \mapsto a_2, m_2 \mapsto a_1 \rbrace$.
According to our formula above, these would have an expected utility of $2$ for both sender and receiver.
% the following expected utility:
% \begin{equation}
% \begin{split}
% EU(\sigma, \rho) & = \sum_{t \in T} U(t, \rho(\sigma(t))) \\
%                  & = U(t_1, \rho(\sigma(t_1))) + U(t_2, \rho(\sigma(t_2))) \\
%                  & = U(t_1, \rho(m_2)) + U(t_2, \rho(m_1)) \\
%                  & = U(t_1, t_1) + U(t_2, t_1) \\
%                  & = (1,1) + (1,1) \\
%                  & = (2,2)
% \end{split}
% \end{equation}
% Based on this, we could also create a matrix of expected utilities for this example as follows:
% \begin{center}
% \begin{tabular}{r|c|c|c|c|}
% \multicolumn{1}{r}{}
%  & \multicolumn{1}{c}{$m_1 \mapsto a_1$}
%  & \multicolumn{1}{c}{$m_1 \mapsto a_1$}
%  & \multicolumn{1}{c}{$m_1 \mapsto a_2$}
%  & \multicolumn{1}{c}{$m_1 \mapsto a_2$} \\
% \multicolumn{1}{r}{}
%  & \multicolumn{1}{c}{$m_2 \mapsto a_1$}
%  & \multicolumn{1}{c}{$m_2 \mapsto a_2$}
%  & \multicolumn{1}{c}{$m_2 \mapsto a_1$}
%  & \multicolumn{1}{c}{$m_2 \mapsto a_2$} \\ \cline{2-5}
%    $t_1 \mapsto m_1, t_2 \mapsto m_1$ & $1,1$ & $1,1$ & $1,1$ & $1,1$ \\ \cline{2-5}
%    $t_1 \mapsto m_1, t_2 \mapsto m_2$ & $1,1$ & $2,2$ & $0,0$ & $1,1$ \\ \cline{2-5}
%    $t_1 \mapsto m_2, t_2 \mapsto m_1$ & $1,1$ & $0,0$ & $2,2$ & $1,1$ \\ \cline{2-5}
%    $t_1 \mapsto m_2, t_2 \mapsto m_2$ & $1,1$ & $1,1$ & $1,1$ & $1,1$ \\ \cline{2-5}
% \end{tabular}
% \end{center}
They also represent one of the two stable conventions in this game, the other being the pair of strategies $\sigma = \lbrace t_1 \mapsto m_1, t_2 \mapsto m_2 \rbrace$ and $\rho = \lbrace m_1 \mapsto a_1, m_2 \mapsto a_2 \rbrace$.
Conventions of this kind in a signaling problem are what Lewis calls \emph{signaling systems}.

The approach adumbrated so far is not, nor does it attempt to be, a full-blown theory of meaning like the classical picture.
However, we can already see it attempts to address the study of meaning from a very different angle.
There is a focus, not on meaning as a kind of correspondence, but rather on the use of signals for specific purposes.
There is also no appeal to truth as a guide for our inquiry.
In this scenario, agents are assumed to possess and exercise rationality.
It is not, however, the same rationality as we find in the classical picture, \emph{i.e.}~the willingness to abide by truth; it is rather a more pragmatic notion of rationality as utility maximization.
One advantage is that we are no longer tied to the need to explain how language hooks on to the world, since this is not relevant on this account.
We are merely trying to see how agents can use signals to cope with the world and achieve their purposes.
On the way, we will better understand meaning, not by trying to say what meaning is, but by seeing how language works.

Brian Skyrms~\parencite*[80--104]{skyrms_evolution_1996}, however, identifies some problems with the story so far.
Lewis' account of the stability of conventions rests on what could be considered strong demands for there to be the required common knowledge between the players involved.
Namely, there needs to be a state of affairs that indicates to everyone involved that a certain regularity will hold, as well as ``mutual ascription of some common inductive standards and background information, rationality, mutual ascription of rationality, and so on''~\parencite*[56--57]{lewis_convention_1969}.
These requirements can seem excessive, even more so if we consider how simple signaling systems are, compared to human languages.
The models were introduced in order to help explain how language as a conventional system could get off the ground without any prior sort of agreement.
If we think of the origins of language in the history of mankind, it seems implausible to assume a high degree of rationality of the agents that started making use of primordial signaling systems that (hypothetically) evolved into languages.
Also, communication through simple message exchange is something that almost all animals do, from monkeys with their calls, to birds with their songs, to bees with their dances, to ants with their pheromone trails.
A plausible account of the origin of language should first explain how signaling systems like those could get started without a great deal of rationality from the part of the agents involved.

In order to address this problem, Skyrms proposes we study signaling problems in evolutionary terms.
Rather than imagining, as Lewis does, rational agents making conscious decisions in possession of knowledge of the game and expectations of the behavior of other agents, we can imagine a simpler scenario inspired by biological evolution: there is a population of agents with biologically hardwired behaviors for engaging in interactions characteristic of a signaling problem; utility does not represent preference, but rather fitness for survival and reproduction; the make-up of the population evolves based on the relative fitness of the strategies represented in the population.
Such a setup attempts to capture the main features of natural selection: in a diverse population, agents with more successful strategies thrive, while agents with less fit strategies die off.
Although the inspiration for this scenario is biological evolution, similar things could be said~\parencite[\emph{e.g.}][]{dawkins_selfish_1978,boyd_culture_1985} about how ideas spread in a population of agents who can adopt or abandon them depending on how successful they appear to be.
The principles can be captured in a formal model that abstracts away from the interpretations: the replicator dynamics.
The only thing relevant to this equation are the relative proportions of strategies in a given population and the utility function.
Using it, one can compute which strategies evolve under which conditions.

The characterization of signaling problems in terms of evolutionary game theory allow us to explain why certain equilibria come to be and how.
Not only can we better understand why signaling systems are stable even without any assumptions of rationality, but we can also map out which initial conditions drive the system towards which equilibria and which don't\todo{mention something about phase portraits}.
In a simple case like the example discussed above, an evolutionary process always drives the system into a state where one signaling system takes over the whole population; which one of the possible two will depend on their relative proportions in the original population.
More complex signaling problems may have different evolutionary outcomes, sometimes surprising ones.
Skyrms~\parencite*{skyrms_signals_2010} gives an overview of different topics studied using signaling games, including expansions of the framework itself (for example, considering other dynamics beyond the replicator equation), exploration of other factors that impact the evolution of signaling (for example, how agents are interconnected), or variations on the signaling problem and its basic assumptions (for example, explanations of deception).

Skyrms' evolutionary game theory approach not only gives more plausible grounds to support Lewis' discussion of convention, but it also accomplishes an important conceptual change, in that it moves most of the theory and mathematical formalisms to the descriptive side of the investigation.
Utility represents how the modeler views the signaling problem and understands the relative advantages or disadvantages of different possible strategy combinations.
Dynamics describe how strategies can evolve when driven by mechanisms of utility maximization.
Focus is put on understanding how various ingredients to the model interact, and which results they produce, not on metaphysical concerns.
While the general framework manages to abstract some details away from the formalization, it does leave room for them, especially when it comes to the dynamics.
We already mentioned the replicator equation that can be seen as representing biological or cultural evolution, but one can also use dynamics inspired on learning mechanisms, or even ones assuming a high degree of knowledge of the game and other players.\todo{give examples and references?}
This range of options goes hand in hand with assumptions of rationality, from no assumptions in a biologically-inspired setting, to a certain degree of rationality captured by a learning method, to higher level rationality and even game-theoretical reasoning.

\todo[inline]{How about vagueness?}



% One interesting aspect of these signaling systems is that, if we were to observe the repeated interactions of sender and receiver according to those strategies from a bird's eye view, we would be inclined to say that the signals are meaningful.
% For example, if we consider the latter example, a sender would always send $m_1$ when state of affairs $t_1$ obtains to which the receiver would always respond with action $a_1$ which maximizes both players' payoff.
% It is as if $m_1$ means either ``$t_1$ obtains'' or ``do $a_1$''.
%


\section{Signaling models of vagueness}
\label{sec:vague-signaling}


\printbibliography

\end{document}
