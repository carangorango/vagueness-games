\documentclass{article}
\usepackage{todonotes}
\usepackage{a4wide}
\usepackage[natbib=true,style=authoryear-comp,backend=biber,doi=false,url=false]{biblatex}
\bibliography{paper}

\begin{document}

\title{Natural sources of vagueness and their implications}
\author{Michael Franke \and Jos\'e Pedro Correia}
\date{}

\maketitle

\begin{abstract}
A vexing puzzle about vagueness, rationality and evolution runs, in crude abbreviation, as follows: vague language use is demonstrably suboptimal if the goal is efficient precise and cooperative information transmission; hence rational deliberation or evolutionary selection should, under this assumed goal, eradicate vagueness from language use.
Since vagueness is persistent in all human languages, something has to give.
In this paper, we investigate a number of reasons why and mechanisms how vagueness may come into the picture in formal models of rational or evolutionary optimal signaling.
We show how uncertainty about not only the linguistic practices of others, but also about the world itself can lead to vagueness, and how, given vagueness, natural linguistic practices are likely to create more reason for vagueness.
We explore the consequences of these reasons and mechanism for a notion of meaning and a notion of language.
\end{abstract}

\tableofcontents

\section{Vaguenes}
\label{sec:vagueness}

\section{Paradigm shift}
\label{sec:Wittgenstein-and-signaling}

\section{Signaling models of vagueness}
\label{sec:vague-signaling}


\printbibliography

\end{document}
