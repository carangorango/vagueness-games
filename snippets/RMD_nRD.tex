\documentclass[fleqn,reqno,10pt]{article}

\usepackage{../helpers/myarticlestyledefault}

\renewcommand{\Smixed}{\ensuremath{\mathrm{\mathbf{s}}}}
\renewcommand{\Rmixed}{\ensuremath{\mathrm{\mathbf{r}}}}

\title{Replicator-mutator and noise-perturbed replicator dynamics}
\author{}
\date{}

\begin{document}
\maketitle


\section{Preliminaries}

\paragraph{Notation.} Fix a signaling game with states $\States$,
messages $\Messgs$ and acts $\Acts$. % Let $\Util_{\sen,\rec} \mycolon
% \States \times \Messgs \times \Acts \rightarrow \mathds{R}$ be the
% player's utility function.

Pure sender (receiver) strategies are functions $\Spure \in
\Messgs^\States$ ($\Rpure \in \Acts^\Messgs$). Mixed sender (receiver)
strategies are functions $\Smixed \in \Delta(\Messgs^\States)$
($\Rmixed \in (\Acts^\Messgs)$). The latter give the relative
population frequencies of the former. We write $\Smixed_i$ for the
frequency $\Smixed(\Spure_i)$ of pure strategy $\Spure_i$ (also for
the receiver).



Let the sender's (receiver's) behavioral strategies be functions in
$\Sstrat \in \Delta(\Messgs)^\States$ ($\Rstrat \in
\Delta(\Acts)^\Messgs$). Every mixed strategy $\Smixed$ converts to a
unique behavioral strategy defined by:
\begin{align*}
  \Sstrat(\messg \probbar \state) = \sum_{\Spure(\state) = \messg} \Smixed(\Spure)\,.
\end{align*} 
Let $G$ be this mapping from mixed to behavioral strategies. Notice
that $G$ is \emph{not} an injection, as many mixed strategies map onto
the same behavioral strategy.

For behavioral strategies, let $\EU(\messg, \state, \Rstrat)$ and
$\EU(\act, \messg, \Sstrat)$ be the players' expected utilities of
each action in each choice point as usual. For mixed strategies we
define the players' fitness in the usual manner. Let $F_i^{\Rmixed}$
be $\Spure_i$'s fitness given $\Rmixed$ and $F_i^{\Smixed}$ be
$\Rpure_i$'s fitness given $\Smixed$. Then $\Phi(\Smixed,\Rmixed) =
\sum_{k} \Smixed_k \cdot F_k^{\Rmixed}$ is the average fitness in the
sender population and $\Phi(\Rmixed,\Smixed) = \sum_{k} \Rmixed_k
\cdot F_k^{\Smixed}$ the average fitness in the receiver population.

\paragraph{Replicator dynamics in pure strategies.} The two-population
(non-payoff adjusted) continuous replicator dynamics in pure
strategies is defined as:
\begin{align*}
  \dot{\Smixed_i} & = \Smixed_i \cdot \left ( F_i^{\Rmixed} -
  \Phi(\Smixed,\Rmixed) \right ) &   \dot{\Rmixed_i} &  = \Rmixed_i \cdot \left ( F_i^{\Smixed} -
  \Phi(\Rmixed,\Smixed) \right ) \,.
\end{align*}
The discrete time version is given by: 
\begin{align*}
  \Smixed_i' & = \frac{\Smixed_i \cdot
  F_i^{\Rmixed}}{ \Phi(\Smixed,\Rmixed)} &     \Rmixed_i' & = \frac{\Rmixed_i \cdot
  F_i^{\Smixed}}{ \Phi(\Rmixed,\Smixed)} \,.
\end{align*}


\paragraph{Replicator-mutator equation.} Let $Q$ be a row-stochastic
mutation matrix where $Q_{ji}$ gives the probability that pure sender
strategy $\Spure_j$ mutates into $\Spure_i$. Similarly, let $R$ be a
row-stochastic mutation matrix where $R_{ji}$ gives the probability
that pure receiver strategy $\Rpure_j$ mutates into $\Rpure_i$.

The two-population (non-payoff adjusted) continuous replicator-mutator
dynamics, is then given by::
\begin{align*}
  \dot{\Smixed_i} & = \sum_{j}  Q_{ji} \cdot \Smixed_j
    \cdot F_j^{\Rmixed} - \Smixed_i \cdot \Phi(\Smixed,\Rmixed) &
    \dot{\Rmixed_i} & = \sum_{j}  R_{ji} \cdot \Rmixed_j
    \cdot F_j^{\Smixed} - \Rmixed_i \cdot \Phi(\Rmixed,\Smixed) \,.
\end{align*}
\begin{claim} A discrete time version of the above replicator mutator
  dynamics is:
  \begin{align*}
    \Smixed_i' & = \sum_{j} Q_{ji} \frac{\Smixed_j \cdot
      F_j^{\Rmixed}}{ \Phi(\Smixed,\Rmixed)} & \Rmixed_i' & = \sum_{j}
    R_{ji} \frac{\Rmixed_j \cdot F_j^{\Smixed}}{
      \Phi(\Rmixed,\Smixed)}\,.
  \end{align*}
\end{claim}

\begin{proof}
  It suffices to check either sender or receiver part. Focusing on the
  former, we need to show that the continuous version is derivable
  from the discrete version if the discrete update steps get
  infinitely small so that:
  \begin{align*}
    \dot{\Smixed_i} & = \Smixed_i' - \Smixed_i = \sum_{j} Q_{ji}
    \frac{\Smixed_j \cdot F_j^{\Rmixed}}{ \Phi(\Smixed,\Rmixed)} -
    \Smixed_i = \sum_{j} \Smixed_j \left ( \frac{ Q_{ji} \cdot
        F_j^{\Rmixed}}{ \Phi(\Smixed,\Rmixed)} - \Smixed_i \right ) \\
    & = \sum_{j} \Smixed_j \left ( \frac{ Q_{ji} \cdot
        F_j^{\Rmixed} - \Smixed_i \cdot \Phi(\Smixed,\Rmixed)}{ \Phi(\Smixed,\Rmixed)}  \right )
  \end{align*}
  As $\Phi(\Smixed,\Rmixed)$ is a constant rate of change, we can drop
  it to derive the non-payoff adjusted continuous version above,
  since:
  \begin{align*}
    & \sum_{j} \Smixed_j \left ( Q_{ji} \cdot
        F_j^{\Rmixed} - \Smixed_i \cdot \Phi(\Smixed,\Rmixed)  \right
      ) =     \sum_{j} \Smixed_j  Q_{ji} \cdot
        F_j^{\Rmixed} - \sum_{j} \Smixed_j \cdot \Smixed_i \cdot
        \Phi(\Smixed,\Rmixed) \\
       = &    \sum_{j} \Smixed_j  Q_{ji} \cdot
        F_j^{\Rmixed} - \Smixed_i \cdot
        \Phi(\Smixed,\Rmixed) 
  \end{align*}
\end{proof}

The discrete time replicator-mutator dynamics has a nice sequential
update property: first we compute the fitness-driven change according
to the standard replicator dynamics; then we compute the perturbation
from mutation.

\section{Noise-perturbed replicator dynamics}

We look at signaling games with $\States = \Acts$ and fix a confusion
matrix $C$, which is a row-stochastic matrix whose elements $C_{ij}$
give the probability that $\state_i$ is perceived as
$\state_j$. Define players average utility at each choice point given
the opponent's strategy as:
\begin{align*}
  \Phi(\state,\Rstrat) & = \sum_{\messg} \Sstrat(\messg \probbar \state) \cdot
\EU(\messg,\state,\Rstrat) &
\Phi(\messg,\Sstrat) & = \sum_{\state} \Rstrat(\state \probbar \messg)
\cdot \EU(\state,\messg,\Sstrat)\,.
\end{align*}
The discrete noise-perturbed replicator dynamics on behavioral
strategies proposed by \citet{Correia2013:The-Bivalent-Tr} is:
\begin{align*}
  \Sstrat'(\messg \probbar \state_i) & = \sum_{j} C_{ji} \cdot
  \frac{\Sstrat(\messg \probbar \state_j) \cdot
    \EU(\messg,\state_j,\Rstrat)} {\Phi(\state_j,\Rstrat)} \\
    \Rstrat'(\state_i \probbar \messg) & = \sum_{j} C_{ji} \cdot
  \frac{\Rstrat(\state_j \probbar \messg) \cdot
    \EU(\state_j,\messg,\Sstrat)} {\Phi(\messg,\Sstrat)}  \,.
\end{align*}
Notice that this update is also sequential: first we calculate an
update according to the standard replicator dynamics (in behavioral
strategies), then we compute the perturbation from perceptual noise.


\section{Exploring the relation}

We will show that the noise-perturbed replicator dynamics defined
above is the behavioral-strategy analogue to the replicator-mutator
dynamics when the only source of mutation is perceptual confusion of
states. Since, in general, the replicator dynamics in two-populations
is not equivalent in its formulations for behavioral and mixed
strategies \citep{Cressman2003:Evolutionary-Dy}, we abstract from the
dynamics and look only at the effect of mutation and
noise-perturbation. This is justified given the above mentioned
sequential nature of both dynamics.

Let the mutation $Q(\Smixed)$ ($R(\Rmixed)$) of a mixed sender
(receiver) strategy be another mixed sender (receiver) strategy
defined by:
\begin{align*}
  Q(\Smixed)_i & =  \sum_j  \Smixed_j \cdot
  Q_{ji} &   R(\Rmixed)_i & =  \sum_{j}  \Rmixed_j \cdot
  R_{ji} \,.
\end{align*}
We would like to compare this to the noise perturbation $C(\Sstrat)$
($C(\Rstrat)$) of behavioral strategy $\Sstrat$ ($\Rstrat)$, which is
another behavioral strategy given by:
\begin{align*}
  C(\Sstrat)(\messg \probbar \state_i) & = \sum_{j} C_{ji} \cdot
  \Sstrat(\messg \probbar \state_j) & C(\Rstrat)(\state_i \probbar
  \messg) & = \sum_{j} C_{ji} \cdot \Rstrat(\state_j \probbar \messg)
  \,.
\end{align*}

We hypothesize that a confusion matrix $C$ should give rise to a
unique mutation matrix $Q^C$ so that whenever $G(\Smixed) = \Sstrat$
we also have $G(Q^C(\Smixed)) = C(\Sstrat)$. Similarly, for the
receiver.

\paragraph{Confusion-based mutations.} There are natural conversions
of $C$ into $Q^C$ and $R^C$. The case for the receiver is easier, so
we start with that.

The probability that $\Rpure$ mutates into $\Rpure'$ is the product of
the probabilities for each $\messg$ that $\Rpure(\messg)$ is perceived
as $\Rpure'(\messg)$. Abusing notation, by refer to the index of
$\Rpure(\messg)$ and $\Rpure'(\messg)$ with $\Rpure(\messg)$ and
$\Rpure'(\messg)$ directly, we define:
\begin{align*}
  R^C_{ji} = \prod_{\messg} C_{\Rpure(\messg)\Rpure'(\messg)}\,.
\end{align*}

Now look at the sender. If $\Spure_j(\state_k)=\messg$ and
$\Spure_i(\state_k)=\messg'$, then the probability of confusion at
state $\state_k$ is given by the chance that $\state_k$ is perceived
as a state $\state_l$ which $\Spure_j$ would map onto $\messg'$. The
overal probability that $\Spure_j$ mutates into $\Sstrat_i$ is the
product of these chances for all states. So, define:
\begin{align*}
  Q^C_{ji} = \prod_{\state_k} \sum_{\state_l \in
    \Spure_j^{-1}(\Spure_i(\state_k))} C_{kl}\,.
\end{align*}

For example, consider a signaling game with two states and two
messages. Let the confusion matrix be:
\begin{align*}
  C=
  \begin{pmatrix}
    .8 & .2 \\
    .2 & .8 
  \end{pmatrix}\,.
\end{align*}
The resulting mutation matrices are:
\begin{align*}
S^C & = \bordermatrix{ ~ & 11 & 12 & 21 & 22 \cr
                      11 & 1 & 0 & 0 & 0 \cr
                      12 & .16 & .64 & .04 & .16 \cr
                      21 & .16 & .04 & .64 & .16 \cr
                      22 & 0 & 0 & 0 & 1 \cr}
&                    
  R^C & = \bordermatrix{ ~ & 11 & 12 & 21 & 22 \cr
                      11 & .64 & .16 & .16 & .04 \cr
                      12 & .16 & .64 & .04 & .16 \cr
                      21 & .16 & .04 & .64 & .16 \cr
                      22 & .04 & .12 & .12 & .64 \cr}\,.
\end{align*}
Here, a pair like $21$, for example, refers to a pure sender strategy
with $\Spure(\state_1=\messg_2$ and $\Spure(\state_2) =
\messg_1$. Similarly for the receiver.

\medskip

\todo[inline]{necessary to prove that mutation matrices are row-stochastic?}

\todo[inline]{maybe worthwhile to think about general properties of
  the mutation matrices that ensue from perceptual confusion?}

\medskip

\begin{theorem}
  \label{thm:sender-eq}
  If $G(\Smixed) = \Sstrat$, then $G(Q^C(\Smixed)) = C(\Sstrat)$.
\end{theorem}




\begin{proof}
  Fix $\Smixed$ and $\Sstrat = G(\Smixed)$. Look first at the rhs of
  the consequent:
  \begin{align*}
    C(\Sstrat)(\messg_y \probbar \state_x) & =  \sum_{\state_l} C_{xl}
    \cdot \Sstrat(\messg_y \probbar \state_l) \\
    & =  \sum_{\state_l} C_{xl}
    \cdot  \sum_{\Spure(\state_l) = \messg_y} \Smixed(\Spure) \\
    & = \sum_{\state_l}
    \sum_{\Spure(\state_l) = \messg_y} \Smixed(\Spure) \cdot C_{xl}\,.
  \end{align*}
  Next consider the lhs of the consequent:
  \begin{align*}
    G(Q^C(\Smixed))(\messg_y \probbar \state_x) & =
    \sum_{\Spure_i(\state_x)=\messg_y} Q^C(\Smixed)(\Spure_i) \\
    & = \sum_{\Spure_i(\state_x)=\messg_y} \sum_{\Spure_j}
    \Smixed(\Spure_j) \cdot Q^C_{ji} \\
    & = \sum_{\Spure_i(\state_x)=\messg_y} \sum_{\Spure_j}
    \Smixed(\Spure_j) \cdot \prod_{\state_l} \sum_{\state_m \in
      \Spure_j^{-1}(\Spure_i(\state_l))} C_{lm} \\
    & = \sum_{\Spure_j} \Smixed(\Spure_j) \cdot
    \sum_{\Spure_i(\state_x)=\messg_y} \prod_{\state_l}
    \sum_{\state_m \in \Spure_j^{-1}(\Spure_i(\state_l))} C_{lm}
  \end{align*}
  To simplify this further we look at a fixed $\Spure_j$ and consider
  the term: 
  \begin{align}
    \label{eq:term}
    \sum_{\Spure_i(\state_x)=\messg_y} \prod_{\state_l} \sum_{\state_m
      \in \Spure_j^{-1}(\Spure_i(\state_l))} C_{lm}\,.
  \end{align}
  Let $Y$ be the row-stochastic matrix with $Y_{kl} = \sum_{\state_m
    \in \Spure_j^{-1}(\messg_l)} C_{km}$. Every pure sender strategy
  maps each state $\state_k$ onto exactly one $Y_{kl}$. If we quantify
  over all pure strategies, we essentially look at each such
  mapping. Term (\ref{eq:term}) above quantifies over all pure
  strategies that map $\state_k$ onto $\messg_y$. The above term then
  sums over all products whose factors are tuples in $\times_{k>2}
  \set{y \setbar \exists l \mycolon y = Y_{kl}}$. So term
  (\ref{eq:term}) expands to (where $e = \card{\States}$ and $d=\card{\Messgs}$):
  \begin{align*}
    & (Y_{11} \cdot Y_{21} \cdot Y_{31} \cdot \ldots \cdot Y_{e1}) +
    (Y_{11} \cdot Y_{21} \cdot Y_{31} \cdot \ldots \cdot Y_{e2}) + 
    \dots \\
    & + (Y_{11} \cdot Y_{2d} \cdot
    Y_{3d} \cdot \ldots \cdot
    Y_{ed})\\
    = &    
  \end{align*}
  But since $Y$ is a row-stochastic matrix, this simplifies to
  $Y_{xy}$. Continuing the derivation with this:
  \begin{align*}
    G(Q^C(\Smixed))(\messg_y \probbar \state_x) 
    & = \sum_{\Spure_j} \Smixed(\Spure_j) \cdot
    \sum_{\state_l \in \Spure_j^{-1}(\messg_y)} C_{xl} \\
    & = \sum_{\Spure_j} \sum_{\state_l \in \Spure_j^{-1}(\messg_y)} \Smixed(\Spure_j) \cdot
     C_{xl} \\
     & = \sum_{\state_l}
    \sum_{\Spure(\state_l) = \messg_y} \Smixed(\Spure) \cdot C_{xl}\,.
  \end{align*}

\end{proof}

\begin{theorem}
  If $G(\Rmixed) = \Rstrat$, then $G(R^C(\Rmixed)) = C(\Rstrat)$.
\end{theorem}

\begin{proof}
  Fix $\Rmixed$ and $\Rstrat = G(\Rmixed)$. The rhs of the consequent
  expands to:
  \begin{align*}
    C(\Rstrat)(\state_x \probbar \messg_y) & = \sum_{\state_j}
    \sum_{i \in \set{k \setbar \Rpure_k(\messg_y) = \state_x}}
    \Rmixed_i \cdot C_{jx} \\
    & = \sum_{\Rpure_i} \sum_{j \in \set{k \setbar \Rpure_i(\messg_y) = \state_j}}
    \Rmixed_i \cdot C_{jx} \\
    & = \sum_{\Rpure_i} \Rmixed_i \cdot C_{\Rpure_i(\messg_y)x} \,.
  \end{align*}
  The rhs expands to:
  \begin{align*}
    G(R^C(\Rmixed))(\state_x \probbar \messg_y) & = \sum_{i}
    \Rmixed_i \cdot \sum_{j \in \set{j \setbar \Rpure_k(\messg_y) =
        \state_x}} 
    \prod_{\messg} C_{\Rstrat_i(\messg)\Rstrat_j(\messg)} \\
    & = \sum_{i}
    \Rmixed_i \cdot ( (C_{\Rpure_i(\messg_1)1} \cdot
      C_{\Rpure_i(\messg_2)1} \cdot \ldots +
      C_{\Rpure_i(\messg_d)1})  \\
      & + \quad (C_{\Rpure_i(\messg_1)1} \cdot
      C_{\Rpure_i(\messg_2)1} \cdot \ldots \cdot
      C_{\Rpure_i(\messg_d)2}) + \ldots  \\
      & + \quad (C_{\Rpure_i(\messg_1)1} \cdot
      C_{\Rpure_i(\messg_2)e} \cdot \ldots \cdot
      C_{\Rpure_i(\messg_d)e}) + \ldots ))
  \end{align*}
  If we assuming without loss of generality that $x=y=1$, then if $d =
  \card{\Messgs}$, we see that:
  \begin{align*}
    \sum_{j \in \set{j \setbar \Rpure_k(\messg_y) = \state_x}}
    \prod_{\messg} C_{\Rstrat_i(\messg)\Rstrat_j(\messg)} = \sum_{i}
    \Rmixed_i \cdot
  \end{align*}
\end{proof}

\printbibliography[heading=bibintoc]

\end{document}
