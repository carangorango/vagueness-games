\section{Conclusions}
\label{sec:conclusions}

Vagueness is a pervasive but seemingly non-disruptive feature of
natural communication and classification systems. From an evolutionary
point of view, the challenge arises to explain how vagueness can
persist under selective pressure for precision. This is foremost a
technical challenge, probing for the possibility of integrating into a
consistent model forces that lead to vagueness with forces that lead
to efficient information transfer. This paper proposed one such model
in the context of sim-max games and explored some of its consequences.

We introduced the replicator diffusion dynamic as a special case of
the established replicator mutator dynamic. It is a special case in
the sense that the \rdd incorporates a particular kind of mutation
that is based on the similarity of states (as defined in the sim-max
games). More concretely, we started from the idea that similar states
are more easily confused than less similar states. To formalize this
idea, we assumed that there is a confusion matrix $C$, where $C_{ij}$
is the probability that a state \mystate{i} is realized as a possibly
different state \mystate{j}. Based on this, we showed two things. For
one, we showed how the discrete-time formulation of the replicator
dynamic can be easily adapted to integrate diffusion of behavior due
to confusion of states. This yielded the \rdd in its basic
discrete-time formulation for behavioral strategies. For another, we
also showed that there exists a translation of the confusion matrix
$C$ into suitable mutation matrices $Q_C$ and $R_C$ such that the
effect of former and latter are effectively equivalent, although $C$
applies to behavioral strategies, while $Q_C$ and $R_C$ apply to mixed
strategies. This shows that the \rdd may be thought of as a special
case of the \rmd.

Based on data from numerical simulations, we demonstrated that the
inclusion of mild levels of stochastic confusability of states does
not undermine the possibility of evolving communicatively successful
signaling strategies at all. On the contrary, strategies that evolved
under mild diffusion induce a highly regular and systematic category
structure that shows the signs of vagueness as desired. There might
even be a higher-order benefit to the presence of imprecision, in that
it can accelerate convergence to optimal categorization, by preventing
evolutionary paths to stay near inefficient non-convex strategies for
a long time. Diffusion thus also unifies and regularizes evolutionary
outcomes in such a way that sub-optimal categorization is
avoided. These results complement similar findings by
\citet{OConnor2013:The-Evolution-o}, obtained for a micro-dynamic
extension of reinforcement learning. The \rdd adds to this a more
general, abstract framework that is not tied to the specific
assumptions of turn-based reinforcement.

Many interesting issues arise in this context that we have not yet
explored. Firstly, we concentrated on the \rdd in behavioral
strategies, because numerical simulations of these are much less
complex and unwieldy than of mixed strategies or analyses of the
symmetrized game. But we also noted that for especially biological
interpretations of the dynamic, symmetrizing would be conceptually
more plausible. It remains to be seen whether there are substantial
differences between the symmetrized version of the \rdd and the
behavioral strategy version that we have explored here.

Secondly, we have only explored the effect of a single confusion
matrix that affected both sender and receiver roles. Obviously, we
could allow different roles to be affected differently by
confusability of states. We might hypothesize that for the unifying
and regularizing effects of diffusion, it is less important that the
receiver's strategies are diffused, as long as the sender's are. But,
again, these issues must wait for further exploration. Connected to
that is the obvious extension that also allows for stochastic
confusability of signals.

Further promising extensions of the work presented here include
investigating higher-dimensional state spaces, possibly with less
uniform, more psychologically real similarity metrics, such as the
three-dimensional color space. Also, to compare the work presented
here better to other work on efficient categorization
\citep[e.g.][]{Mohlin2014:Optimal-Categor}, it would be worthwhile
exploring the possibility of signal innovation
\citep[e.g.][]{McKenzie-AlexanderSkymrs2012:Inventing-New-S} also for
sim-max games. This way, the game dynamics would ideally lead to an
optimal number of categories, not only an optimal shape of a
predefined number of categories.

%%% Local Variables: 
%%% mode: latex
%%% TeX-master: "paper"
%%% TeX-PDF-mode: t
%%% End:



