\documentclass[fleqn,reqno,10pt]{article}


%========================================
% Packages
%========================================

\usepackage[]{../helpers/mypackages}
%\usepackage[natbib=true,style=authoryear-comp,backend=bibtex,doi=false,url=false]{biblatex}
\bibliography{../helpers/MyRefGlobal}
\bibliography{paper}
\usepackage{../helpers/myenvironments}
\usepackage{../helpers/mycommands}
\usepackage{todonotes}
\usepackage{subcaption}

%========================================
% Standard Layout
%========================================

% Itemize
\renewcommand{\labelitemi}{\large{$\mathbf{\cdot}$}}    % itemize symbols
\renewcommand{\labelitemii}{\large{$\mathbf{\cdot}$}}
\renewcommand{\labelitemiii}{\large{$\mathbf{\cdot}$}}
\renewcommand{\labelitemiv}{\large{$\mathbf{\cdot}$}}
% Description
\renewcommand{\descriptionlabel}[1]{\hspace\labelsep\textsc{#1}}

% Figure Captions
\usepackage{caption} % use corresponding myfiguresize!
\setlength{\captionmargin}{20pt}
\renewcommand{\captionfont}{\small}
\setlength{\belowcaptionskip}{7pt} % standard is 0pt

%========================================
% Additional layout & commands
%========================================


\renewcommand{\Smixed}{\ensuremath{\mathrm{\mathbf{s}}}}
\renewcommand{\Rmixed}{\ensuremath{\mathrm{\mathbf{r}}}}

% Annotations
\newcommand{\mytodo}[2]{\todo[inline,color=yellow,author=#1]{#2}}
\newcommand{\question}[2]{\todo[inline,color=blue,author=#1]{#2}}
\newcommand{\answer}[2]{\todo[inline,color=green,author=#1]{#2}}


\title{Vagueness, Noise, and Communication}
\author{Jos\'e Pedro Correia and Michael Franke}
\date{}

\begin{document}
\maketitle

\begin{abstract}
  Signaling games have been garnering a lot of attention as models for
  studying natural language meaning.  One important aspect that is
  nevertheless typically missing from most models in the literature is
  the possibility for vagueness as a feature of successful languages.
  Recently, some models have been proposed to explicitly address this
  limitation.  In this paper we explore one such model and establish a
  link between the dynamics in the model and the replicator-mutator
  equations.  Furthermore, we study the long term behavior of the
  model using computer simulations.  The results indicate that vague
  languages, despite having lower expected utility and informativity,
  are faster to converge to an equilibrium than non-vague ones.  We
  argue that this could be an important advantage that contributes to
  the pervasiveness of vagueness in natural language.
\end{abstract}

\section{Introduction}

Many of our concepts and words are vague. A vague category knows clear
cases that fall under it, clear cases that do not, and also so-called
borderline cases in-between that neither clearly apply, nor clearly
not apply, but might themselves not be equal in terms of how well they
represent the category in question. Vagueness does not seem to
dramatically affect the success of everyday communication, but it is
troublesome for some of our best theories of language. Especially the
logico-semantic tradition of Frege, Russell and the young Wittgenstein
is challenged by vagueness and the paradoxes that it gives rise to. As
can be expected, many proponents of this tradition have risen to the
challenge \citep{Williamson1994:Vagueness}.

There are other intriguing aspects about vagueness. Linguists, for
example, wonder about the reasons why some expressions are less vague
than others \citep[e.g.][]{Kennedy2007:Vagueness-and-G}. But the
puzzle that we are concerned with here is how vagueness could arise
and be maintained in the first place. This may not seem a particularly
deep issue at first glance, but on closer inspection it is a serious
worry to functionalist accounts that maintain that our concepts and
language use evolved in order to be efficient. Since the existence of
unclear borderline cases seems to entail inefficiency of
categorization or communication, the challenge, succinctly put forward
by \citet{Lipman2009:Why-is-Language}, is to explain how vagueness can
exist despite its obvious and demonstrable functional deficiency under
evolutionary pressure to be optimally expressive.

This problem, call it Lipman's problem, has a conceptual and a
technical side to it. Conceptually, it asks for reasons and general
mechanisms by which we could plausibly conceive of vagueness as
persisting the pressure of evolutionary optimization towards
precision. Such reasons and general mechanisms are, arguably, not hard
to imagine. We will address some presently. But each putative
explanation, no matter how plausible intuitively, must also be checked
for internal consistency and its potential to account for the fact
that we see vagueness as a part of a by-and-large efficient system of
categorization and communication. In other words, not all first-shot
rebuttals of Lipman's problem will do. Denying that there is any
functional pressure on efficient communication, for instance, is
dubious because it leaves the relative efficiency of our communication
with vague words entirely unexplained. In sum, the technical answer to
Lipman's problem is to give, ideally, a model of evolution of
categorization or language use that shows how competing forces that
realize both by-and-large efficient categories and vagueness. 

c

\bigskip





\begin{itemize}
\item introduce sim-max games
\item problem of vagueness \citep{Lipman2009:Why-is-Language}
\item \dots
\end{itemize}

\subsection{Situating our approach}

A number of authors have recently tried to explain why vagueness
evolved, based on considerations why a vague language might be useful
\citep[e.g.][]{Jaegherde-Jaegher2003:A-Game-Theoreti,Deemter2009:Utility-and-Lan,Jaegherde-JaegherRooijvan-Rooij2010:Strategic-Vague,BlumeBoard2013:Intentional-Vag}. In
contrast to these, others have argued that vagueness if a natural
byproduct of limitations in information processing
\citep[e.g.][]{FrankeJager2010:Vagueness-Signa,OConnor2013:The-Evolution-o}. This
paper makes a contribution to the latter line of thought.

The view that vagueness is a natural concomitant of cognitive
limitations of language users has been formalized in a number of
ways. As in other applications of evolutionary game theory, we can
distinguish micro- and macro-level approaches. Micro-level approaches
look at adaptive behavior of individual agents. Usually, changes in
the behavioral dispositions of agents occur after every single
interaction. In contrast, macro-level approaches outline more
abstract, aggregate population-based dynamics. Usually, a macro-level
dynamic captures changes of frequencies of behavioral types in the
population over time. Previous accounts have suggested both micro- and
macro-level dynamics to show how vague signal use can arise in sim-max
games.

As for micro-dynamics, \citet{FrankeJager2010:Vagueness-Signa} show
how limited memory of past interactions can lead to vague signal use,
when averaging over a single agent's behavior over time or over the
momentary behavior of a population of several language
users. \citet{OConnor2013:The-Evolution-o} introduces a variant of
reinforcement learning that includes a low-level form of stimulus
generalization. Agents update their behavior after each round of play
in such a way that similar states to the ones that actually occurred
are subject to behavioral adjustment as well. (Alternative approaches
are discussed in more detail in Section~XYZ, where we can compare them
with the system introduced in this paper.)

\citet{FrankeJager2010:Vagueness-Signa} also consider a macro-level
approach, using the notion of a \emph{quantal response}. A quantal
response function is a probabilistic choice rule that formalizes the
idea that agents make small mistakes when calculating the expected
utility of choice options. In aggregation, these probabilistic
mistakes lead to systematic ``trembles'' that produce vague signal
use. The approach we take here is superficially similar, but there are
crucial divergences. For one, we adopt a dynamic perspective by
looking at the stable states of a suitable population dynamics. For
another, we demonstrate in Section~XYZ that quantal responses can give
rise to counterintuitive predictions. These counterintuitive examples
suggest that vagueness in signaling use is not convincingly explained
by appeal to mistakes in calculating expected utility, but rather, as
we assume here, as the result of confusing similar states of the word.

Against this background, we introduce a new macro-level approach
to the evolution of vague signal use in sim-max games, where the
source of vagueness is the agents' natural disability to sharply
distinguish perceptually distinct states. This applies to both
production and comprehension, i.e., both to the sender as well as to
the receiver role. 

We call our dynamic \textbf{replicator diffusion dynamic} (\rdd),
because it is a special case of the replicator mutator dynamic
(\rmd). It is a special case in the sense that the \rdd incorporates a
special kind of mutation that is based on the similarity of states (as
defined in the sim-max game). More concretely, we start from the idea
that similar states are more easily confused than less similar
states. To formalize this idea, we assume that there is a confusion
matrix $C$, where $C_{ij}$ is the probability that a state \mystate{i}
is realized as a possibly different state \mystate{j}. (There is a
slightly different sense of ``realization'' in the sender and the
receiver side, and we will enlarge on this below.) Based on this, we
show two things. For one, we show how the discrete-time formulation of
the replicator dynamic can be easily adapted to integrate diffusion of
behavior due to confusion of states. This yields the \rdd in its basic
discrete-time formulation. For another, we also show that there exists
a translation of the confusion matrix $C$ into a suitable mutation
matrix $M_C$ such that the effect of $C$ and $M_C$ are ``equivalent''
in a sense to be made precise. This supports the conclusion that the
\rdd can indeed be thought of as a special case of the \rmd.

To understand the workings of the \rdd, we study numerical simulations
\dots

The paper is structured as follows:
\begin{enumerate}
\item replicator diffusion dynamics
\item relation to replicator mutator dynamics
\item simulations of \rdd
\item comparison with other approaches
\item conclusions
\end{enumerate}


%%% Local Variables: 
%%% mode: latex
%%% TeX-master: "paper"
%%% TeX-PDF-mode: t
%%% End:





\section{Replicator diffusion dynamic}

The replicator diffusion dynamic (\rdd) was first introduced by
\citet{Correia2013:The-Bivalent-Tr} as a noise-perturbed variant of
the replicator dynamic (\rd) in behavioral strategies. We begin by
recapitulating the formulation of the \rd in behavioral
strategies. Then we define the replicator diffusion dynamics in
behavioral strategies, following \citet{Correia2013:The-Bivalent-Tr}.
Finally we show that the \rdd is a special case of the \rmd.

\subsection{Replicator dynamic in behavioral strategies}

Fix a signaling game with states $\States$, messages $\Messgs$ and
acts $\Acts$. Let $\Pr(\cdot) \in \Delta{\States}$ be the prior
distribution over states and $\Util_{\sen,\rec} \mycolon \States
\times \Messgs \times \Acts \rightarrow \mathds{R}$ the senders's and
receiver's utility functions. The sender's behavioral strategies are
functions $\Sstrat \in \Delta(\Messgs)^\States$; the receiver's are
functions $\Rstrat \in \Delta(\Acts)^\Messgs$. Define the expected
utility of actions in each choice point are defined as usual:
\begin{align*}
  \EU(\messg \myts ; \myts \state, \Rstrat) & = \sum_{\act \in \Acts}
  \Rstrat(\act \probbar \messg) \cdot U_\sen(\state, \messg, \act) \\
  \EU(\act \myts ; \myts \messg, \Sstrat) & = \sum_{\state \in
    \States} \Pr(\state) \cdot \Sstrat(\messg \probbar \state) \cdot
  U_\rec(\state, \messg, \act) \,.
\end{align*}
The \emph{fitness} of a behavioral strategy at a choice point is the
frequency-weighted average of expected utilities of each choice, given
the opponent's strategy:
\begin{align*}
  \Phi(\state,\Sstrat, \Rstrat) & = \sum_{\messg} \Sstrat(\messg \probbar \state) \cdot
\EU(\messg \myts ; \myts\state,\Rstrat) \\
\Phi(\messg,\Sstrat, \Rstrat) & = \sum_{\act} \Rstrat(\act \probbar \messg)
\cdot \EU(\act \myts ; \myts \messg,\Sstrat)\,.
\end{align*}
The discrete-time replicator dynamic maps current strategies $\Sstrat$
and $\Rstrat$ to future strategies $\RD(\Sstrat)$ and $\RD(\Rstrat)$ in
such a way that changes in frequency are proportional to expected
utilities. For behavioral strategies, the changes take place locally
at each of the agents' choice points:\todo{could be shortened by using
  $\propto$ notation}
\begin{align*}
  \RD(\Sstrat)(\messg \probbar \state_i) & = \frac{\Sstrat(\messg \probbar \state_j) \cdot
    \EU(\messg,\state_j,\Rstrat)} {\Phi(\state_j,\Sstrat, \Rstrat)} \\
    \RD(\Rstrat)(\state_i \probbar \messg) & = \frac{\Rstrat(\state_j \probbar \messg) \cdot
    \EU(\state_j,\messg,\Sstrat)} {\Phi(\messg,\Sstrat, \Rstrat)}  \,.
\end{align*}

\todo[inline]{example? sim-max game 10 states, after so and so many rounds?}

\subsection{Replicator diffusion dynamic in behavioral strategies}

The replicator diffusion dynamic adds probabilistic confusion of
similar states to the replicator dynamics. Fix a sim-max game with
$\States = \Acts$ and a confusion matrix $C$. $C$ is a row-stochastic
matrix whose elements $C_{ij}$ give the probability that $\state_i$ is
realized as $\state_j$. The confusability of states affects senders
and receivers alike, but in slightly different ways (see
Figure~\ref{fig:noise-perturbation-of-strategies}). For the sender,
$C_{ij}$ is the probability that the actual states \mystate{i} is
perceived as \mystate{j}. For the receiver, $C_{ij}$ is the
probability that \mystate{j} is the interpretation that is actually
formed when \mystate{i} is the output of the receiver's strategy.

\begin{figure}
  \centering

    \begin{tikzpicture}[node distance = 2cm, thick]

      \begin{scope}
  
      \node[rectangle, draw=mycol, fill=mycol!25, thick] (actual)
      {actual state $\mystate{i}$};

      \node[rectangle, draw=mycol, fill=mycol!25, thick, below of =
      actual] (perceived) {perceived state $\mystate{j}$};

      \node[rectangle, draw=mycol, fill=mycol!25, thick, below of =
      perceived] (output) {chosen message $\messg$};

      \node[rectangle, thick, below of = output, node distance=1cm]
      (sender) {sender};

      \draw[->] (actual) -> (perceived) node[midway,left] {noise
        $C_{ij}$};

      \draw[->] (perceived) -> (output) node[midway,left] (label)
      {strategy $\Sstrat(\messg \probbar \mystate{j})$};

   
      \begin{pgfonlayer}{background}
        \node [draw=black!50, fill=black!20,fit=(actual) (label)
        (output)] {};
      \end{pgfonlayer}
    \end{scope}


      \begin{scope}[xshift=5cm]
  
      \node[rectangle, draw=mycol, fill=mycol!25, thick] (message)
      {observed message $\messg$};

      \node[rectangle, draw=mycol, fill=mycol!25, thick, below of =
      message] (target) {target interpretation $\mystate{i}$};

      \node[rectangle, draw=mycol, fill=mycol!25, thick, below of =
      target] (realized) {realized interpretation $\mystate{j}$};

      \node[rectangle, thick, below of = realized, node distance=1cm]
      (sender) {receiver};

      \draw[->] (message) -> (target) node[midway,right] (bla) {strategy
        $\Rstrat(\mystate{i} \probbar \messg)$};

      \draw[->] (target) -> (realized) node[midway,right]  {noise $C_{ij}$};

   
      \begin{pgfonlayer}{background}
        \node [draw=black!50, fill=black!20,fit = (message) 
        (realized) (bla)] {};
      \end{pgfonlayer}
    \end{scope}

  \end{tikzpicture}

  \caption{Effect of confusion of states on sender and receiver
    choices.}
  \label{fig:noise-perturbation-of-strategies}
\end{figure}

The aggregate effect of confusion of states on behavioral strategies
can be captured in a function that maps behavioral strategies
$\Sstrat$ and $\Rstrat$ on their noise-perturbed realizations
$C(\Sstrat)$ and $C(\Rstrat)$. The idea is that $\Sstrat$ and
$\Rstrat$ are what, on average, the idealized, confusion-free behavior
would be, while $C(\Sstrat)$ and $C(\Rstrat)$ are behavioral
strategies that describe the agents' actual noise-perturbed
probabilistic behavior. If we conceive of behavioral strategies
$\Sstrat$ and $\Rstrat$ as row-stochastic matrices, the effect of
state confusability is easily captured by matrix multiplication as:
\begin{align*}
  C(\Sstrat) & = C \Sstrat &    C(\Rstrat) & = \Rstrat C\,.
\end{align*}

The discrete-time replicator diffusion dynamic takes the replicator
dynamic as basic, but factors in the confusion of states at each
update step in a sequential update:
\begin{align*}
  \RDD(\Sstrat) & = C(\RD(\Sstrat)) &   \RDD(\Rstrat) & = C(\RD(\Rstrat)) 
\end{align*}
This is equivalent to the following definition, given by
\citet{Correia2013:The-Bivalent-Tr}:
\begin{align*}
  \RDD(\Sstrat)(\messg \probbar \state_i) & = \sum_{j} C_{ij} \cdot
  \frac{\Sstrat(\messg \probbar \state_j) \cdot
    \EU(\messg,\state_j,\Rstrat)} {\Phi(\state_j,\Rstrat)} \\
    \RDD(\Rstrat)(\state_i \probbar \messg) & = \sum_{j} C_{ij} \cdot
  \frac{\Rstrat(\state_j \probbar \messg) \cdot
    \EU(\state_j,\messg,\Sstrat)} {\Phi(\messg,\Sstrat)}  \,.
\end{align*}

The idea behind the sequential definition of the \rdd is that, at each
time step, strategies are gradually optimized along the current
fitness landscape, as described by the \rd, but the realization of
optimized strategies is bound to be noisy, due to confusion of
states. It is obvious that other (discrete-time) evolutionary dynamics
can be subjected to state-confusability in an analogous fashion
\citep{Correia2013:The-Bivalent-Tr}. In the case of the \rd, however,
perturbation by state-confusion has a prominent close relative in the
replicator mutator dynamic.

\subsection{Relation with the replicator mutator dynamic}

The replicator mutator dynamic (\rmd) has been proposed by Martin
Nowak first in the context of signaling game models for the evolution
of grammar
\citep[e.g.][]{KomarovaNiyogi2001:The-Evolutionar,NowakKomarova2001:Evolution-of-Un,Nowak2006:Evolutionary-Dy}. The
relation of the \rmd to other prominent evolutionary dynamics is well
understood \citep{PageNowak2002:Unifying-Evolut}. The \rmd has seen
further fruitful applications in the context of signaling games
\citep{HutteggerSkyrms2010:Evolutionary-Dy}. It is therefore desirable
to relate the \rdd that we propose here to the \rmd.

A direct comparison between \rdd and the \rmd is not possible, because
the usual formulation of the latter is in its continuous-time form,
and based on mixed strategies, not behavioral strategies. We therefore
give a discrete-time formulation of the \rmd in behavioral strategies,
which is independently useful. The relation between \rdd and \rmd will
then be plain to see.

\subsubsection{Replicator mutator dynamic}

Pure sender (receiver) strategies are functions $\Spure \in
\Messgs^\States$ ($\Rpure \in \Acts^\Messgs$). Mixed sender (receiver)
strategies are functions $\Smixed \in \Delta(\Messgs^\States)$
($\Rmixed \in (\Acts^\Messgs)$). The latter give the relative
population frequencies of the former. We write $\Smixed_i$ for the
frequency $\Smixed(\Spure_i)$ of pure strategy $\Spure_i$ (also for
the receiver). For mixed strategies, we define the players' fitness in
the usual manner. Let $F_i^{\Rmixed}$ be $\Spure_i$'s fitness given
$\Rmixed$ and $F_i^{\Smixed}$ be $\Rpure_i$'s fitness given
$\Smixed$. Then $\Phi(\Smixed,\Rmixed) = \sum_{k} \Smixed_k \cdot
F_k^{\Rmixed}$ is the average fitness in the sender population and
$\Phi(\Rmixed,\Smixed) = \sum_{k} \Rmixed_k \cdot F_k^{\Smixed}$ the
average fitness in the receiver population.

Every mixed strategy $\Smixed$ converts to a unique behavioral
strategy defined by:
\begin{align*}
  \Sstrat(\messg \probbar \state) = \sum_{\Spure(\state) = \messg} \Smixed(\Spure)\,.
\end{align*} 
Let $G$ be this mapping from mixed to behavioral strategies. Notice
that $G$ is \emph{not} an injection, as many mixed strategies map onto
the same behavioral strategy. \todo{this should not be here}



\paragraph{Replicator dynamics in pure strategies.} The two-population
(non-payoff adjusted) continuous replicator dynamics in pure
strategies is defined as:
\begin{align*}
  \dot{\Smixed_i} & = \Smixed_i \cdot \left ( F_i^{\Rmixed} -
  \Phi(\Smixed,\Rmixed) \right ) &   \dot{\Rmixed_i} &  = \Rmixed_i \cdot \left ( F_i^{\Smixed} -
  \Phi(\Rmixed,\Smixed) \right ) \,.
\end{align*}
The discrete time version is given by: 
\begin{align*}
  \Smixed_i' & = \frac{\Smixed_i \cdot
  F_i^{\Rmixed}}{ \Phi(\Smixed,\Rmixed)} &     \Rmixed_i' & = \frac{\Rmixed_i \cdot
  F_i^{\Smixed}}{ \Phi(\Rmixed,\Smixed)} \,.
\end{align*}


\paragraph{Replicator-mutator equation.} Let $Q$ be a row-stochastic
mutation matrix where $Q_{ji}$ gives the probability that pure sender
strategy $\Spure_j$ mutates into $\Spure_i$. Similarly, let $R$ be a
row-stochastic mutation matrix where $R_{ji}$ gives the probability
that pure receiver strategy $\Rpure_j$ mutates into $\Rpure_i$.

The two-population (non-payoff adjusted) continuous replicator-mutator
dynamics, is then given by::
\begin{align*}
  \dot{\Smixed_i} & = \sum_{j}  Q_{ji} \cdot \Smixed_j
    \cdot F_j^{\Rmixed} - \Smixed_i \cdot \Phi(\Smixed,\Rmixed) &
    \dot{\Rmixed_i} & = \sum_{j}  R_{ji} \cdot \Rmixed_j
    \cdot F_j^{\Smixed} - \Rmixed_i \cdot \Phi(\Rmixed,\Smixed) \,.
\end{align*}
\begin{claim} A discrete time version of the above replicator mutator
  dynamics is:
  \begin{align*}
    \Smixed_i' & = \sum_{j} Q_{ji} \frac{\Smixed_j \cdot
      F_j^{\Rmixed}}{ \Phi(\Smixed,\Rmixed)} & \Rmixed_i' & = \sum_{j}
    R_{ji} \frac{\Rmixed_j \cdot F_j^{\Smixed}}{
      \Phi(\Rmixed,\Smixed)}\,.
  \end{align*}
\end{claim}

\begin{proof}
  It suffices to check either sender or receiver part. Focusing on the
  former, we need to show that the continuous version is derivable
  from the discrete version if the discrete update steps get
  infinitely small so that:
  \begin{align*}
    \dot{\Smixed_i} & = \Smixed_i' - \Smixed_i = \sum_{j} Q_{ji}
    \frac{\Smixed_j \cdot F_j^{\Rmixed}}{ \Phi(\Smixed,\Rmixed)} -
    \Smixed_i = \sum_{j} \Smixed_j \left ( \frac{ Q_{ji} \cdot
        F_j^{\Rmixed}}{ \Phi(\Smixed,\Rmixed)} - \Smixed_i \right ) \\
    & = \sum_{j} \Smixed_j \left ( \frac{ Q_{ji} \cdot
        F_j^{\Rmixed} - \Smixed_i \cdot \Phi(\Smixed,\Rmixed)}{ \Phi(\Smixed,\Rmixed)}  \right )
  \end{align*}
  As $\Phi(\Smixed,\Rmixed)$ is a constant rate of change, we can drop
  it to derive the non-payoff adjusted continuous version above,
  since:
  \begin{align*}
    & \sum_{j} \Smixed_j \left ( Q_{ji} \cdot
        F_j^{\Rmixed} - \Smixed_i \cdot \Phi(\Smixed,\Rmixed)  \right
      ) =     \sum_{j} \Smixed_j  Q_{ji} \cdot
        F_j^{\Rmixed} - \sum_{j} \Smixed_j \cdot \Smixed_i \cdot
        \Phi(\Smixed,\Rmixed) \\
       = &    \sum_{j} \Smixed_j  Q_{ji} \cdot
        F_j^{\Rmixed} - \Smixed_i \cdot
        \Phi(\Smixed,\Rmixed) 
  \end{align*}
\end{proof}

The discrete time replicator-mutator dynamics has a nice sequential
update property: first we compute the fitness-driven change according
to the standard replicator dynamics; then we compute the perturbation
from mutation.

\subsection{Noise-perturbed replicator dynamics}

We look at signaling games with $\States = \Acts$ and fix a confusion
matrix $C$, which is a row-stochastic matrix whose elements $C_{ij}$
give the probability that $\state_i$ is perceived as
$\state_j$. Define the players' average utility at each choice point given
the opponent's strategy as:
\begin{align*}
  \Phi(\state,\Rstrat) & = \sum_{\messg} \Sstrat(\messg \probbar \state) \cdot
\EU(\messg,\state,\Rstrat) &
\Phi(\messg,\Sstrat) & = \sum_{\state} \Rstrat(\state \probbar \messg)
\cdot \EU(\state,\messg,\Sstrat)\,.
\end{align*}
The discrete noise-perturbed replicator dynamics on behavioral
strategies proposed by \citet{Correia2013:The-Bivalent-Tr} is:
\begin{align*}
  \Sstrat'(\messg \probbar \state_i) & = \sum_{j} C_{ji} \cdot
  \frac{\Sstrat(\messg \probbar \state_j) \cdot
    \EU(\messg,\state_j,\Rstrat)} {\Phi(\state_j,\Rstrat)} \\
    \Rstrat'(\state_i \probbar \messg) & = \sum_{j} C_{ji} \cdot
  \frac{\Rstrat(\state_j \probbar \messg) \cdot
    \EU(\state_j,\messg,\Sstrat)} {\Phi(\messg,\Sstrat)}  \,.
\end{align*}
Notice that this update is also sequential: first we calculate an
update according to the standard replicator dynamics (in behavioral
strategies), then we compute the perturbation from perceptual noise.


\subsection{Exploring the relation}

We will show that the noise-perturbed replicator dynamics defined
above is the behavioral-strategy analogue to the replicator-mutator
dynamics when the only source of mutation is perceptual confusion of
states. Since, in general, the replicator dynamics in two-populations
is not equivalent in its formulations for behavioral and mixed
strategies \citep{Cressman2003:Evolutionary-Dy}, we abstract from the
dynamics and look only at the effect of mutation and
noise-perturbation. This is justified given the above mentioned
sequential nature of both dynamics.

Let the mutation $Q(\Smixed)$ ($R(\Rmixed)$) of a mixed sender
(receiver) strategy be another mixed sender (receiver) strategy
defined by:
\begin{align*}
  Q(\Smixed)_i & =  \sum_j  \Smixed_j \cdot
  Q_{ji} &   R(\Rmixed)_i & =  \sum_{j}  \Rmixed_j \cdot
  R_{ji} \,.
\end{align*}
We would like to compare this to the noise perturbation $C(\Sstrat)$
($C(\Rstrat)$) of behavioral strategy $\Sstrat$ ($\Rstrat)$, which is
another behavioral strategy given by:
\begin{align*}
  C(\Sstrat)(\messg \probbar \state_i) & = \sum_{j} C_{ji} \cdot
  \Sstrat(\messg \probbar \state_j) & C(\Rstrat)(\state_i \probbar
  \messg) & = \sum_{j} C_{ji} \cdot \Rstrat(\state_j \probbar \messg)
  \,.
\end{align*}

We hypothesize that a confusion matrix $C$ should give rise to a
unique mutation matrix $Q^C$ so that whenever $G(\Smixed) = \Sstrat$
we also have $G(Q^C(\Smixed)) = C(\Sstrat)$. Similarly, for the
receiver.

\paragraph{Confusion-based mutations.} There are natural conversions
of $C$ into $Q^C$ and $R^C$. The case for the receiver is easier, so
we start with that.

The probability that $\Rpure$ mutates into $\Rpure'$ is the product of
the probabilities for each $\messg$ that $\Rpure(\messg)$ is perceived
as $\Rpure'(\messg)$. Abusing notation, by refer to the index of
$\Rpure(\messg)$ and $\Rpure'(\messg)$ with $\Rpure(\messg)$ and
$\Rpure'(\messg)$ directly, we define:
\begin{align*}
  R^C_{ji} = \prod_{\messg} C_{\Rpure(\messg)\Rpure'(\messg)}\,.
\end{align*}

Now look at the sender. If $\Spure_j(\state_k)=\messg$ and
$\Spure_i(\state_k)=\messg'$, then the probability of confusion at
state $\state_k$ is given by the chance that $\state_k$ is perceived
as a state $\state_l$ which $\Spure_j$ would map onto $\messg'$. The
overal probability that $\Spure_j$ mutates into $\Sstrat_i$ is the
product of these chances for all states. So, define:
\begin{align*}
  Q^C_{ji} = \prod_{\state_k} \sum_{\state_l \in
    \Spure_j^{-1}(\Spure_i(\state_k))} C_{kl}\,.
\end{align*}

For example, consider a signaling game with two states and two
messages. Let the confusion matrix be:
\begin{align*}
  C=
  \begin{pmatrix}
    .8 & .2 \\
    .2 & .8 
  \end{pmatrix}\,.
\end{align*}
The resulting mutation matrices are:
\begin{align*}
S^C & = \bordermatrix{ ~ & 11 & 12 & 21 & 22 \cr
                      11 & 1 & 0 & 0 & 0 \cr
                      12 & .16 & .64 & .04 & .16 \cr
                      21 & .16 & .04 & .64 & .16 \cr
                      22 & 0 & 0 & 0 & 1 \cr}
&                    
  R^C & = \bordermatrix{ ~ & 11 & 12 & 21 & 22 \cr
                      11 & .64 & .16 & .16 & .04 \cr
                      12 & .16 & .64 & .04 & .16 \cr
                      21 & .16 & .04 & .64 & .16 \cr
                      22 & .04 & .12 & .12 & .64 \cr}\,.
\end{align*}
Here, a pair like $21$, for example, refers to a pure sender strategy
with $\Spure(\state_1=\messg_2$ and $\Spure(\state_2) =
\messg_1$. Similarly for the receiver.

\medskip

\todo[inline]{necessary to prove that mutation matrices are row-stochastic?}

\todo[inline]{maybe worthwhile to think about general properties of
  the mutation matrices that ensue from perceptual confusion?}

\medskip

\begin{theorem}
  \label{thm:sender-eq}
  If $G(\Smixed) = \Sstrat$, then $G(Q^C(\Smixed)) = C(\Sstrat)$.
\end{theorem}


\begin{theorem}
  \label{thm:receiver-eq}
  If $G(\Rmixed) = \Rstrat$, then $G(R^C(\Rmixed)) = C(\Rstrat)$.
\end{theorem}

%%% Local Variables: 
%%% mode: latex
%%% TeX-master: "paper"
%%% TeX-PDF-mode: t
%%% End:





\section{Simulations}

%%% Local Variables: 
%%% mode: latex
%%% TeX-master: "paper"
%%% TeX-PDF-mode: t
%%% End:





\section{Discussion}
\label{sec:discussion}

We have introduced a new variant of the replicator mutator dynamics
that implements specifically stochastic noise in the form of
confusability of similar states. The probabilities of confusing nearby
similar states is easily implemented in behavioral strategies, but can
also be translated into a mutation matrix for pure strategies. This
allows, in principle, several conceptual interpretations of the \rdd,
on which we will briefly elaborate below in
Section~\ref{sec:model-interpretation}.

Next to this technical contribution, the results of our numerical
simulations also advance our understanding of the possibility of
evolving regular and efficient categories despite their (higher-order)
vagueness. Section~\ref{sec:relat-with-prev} zooms in on the relation
with closely related accounts once more, to delineate the present
approach more precisely.

\subsection{Model interpretation}
\label{sec:model-interpretation}

We mentioned in passing in Section~\ref{sec:repl-diff-dynam-1} that
adding diffusion to other discrete-time evolutionary dynamics that
operate on behavioral strategies is entirely straightforward. We
could, for instance, easily diffuse the outcome of a best-response
dynamic at each time step. That would make sense if we thought of
agents as prone to confusing similar states that are otherwise
rational optimizers of behavior. The reason that we chose the
replicator dynamics to combine with diffusion is twofold. A minor
reason is that it makes for a conceptually interesting link with the
replicator mutator dynamics. A more important reason is that the
replicator dynamic is especially versatile and non-committal about
what the exact process of adaption is that is being modelled.

Originally the \rd was introduced as mathematical model of evolution
under asexual reproduction, motivated by concepts from the theory of
natural selection \citep{TaylorJonker1978:Evolutionary-St}. The most
conservative interpretation of the \rdd in our present context is thus
a biological one: we can imagine signaling strategies as innate or
fixed behaviorial tendencies of organisms, steps in the evolutionary
process as successive generations, and selection as capturing the
reproductive advantage of fitter individuals. This interpretation,
strictly speaking, requires formalization in terms of mixed strategies
via the \rmd, and possibly also a symmetrizing of the game, so that
every agent is assumed to have a unique sender and receiver role at
the same time the is bequeathed onto the next generation. Diffusion,
in this context, could be either of two things. It could be
differential mutation probabilities in line with other interpretation
of mutation in the \rmd
\citep[e.g.][]{NowakKomarova2001:Evolution-of-Un,KomarovaNiyogi2001:The-Evolutionar}:
pure strategies are not faithfully reproduced, and mistakes in
bequeathing pure strategies are more likely if they result from the
confusion of similar states. Another possible biological
interpretation of the \rdd, is that inheritance is faithful, but
strategies are noisily realized. Strategies are not necessarily
selected for what they are, but rather for how they are realized, once
noise is factored in. 

But the replicator dynamic is not only a model of biological
evolution. It can also be interpreted as a high-level description of
the likely development of other behavioral adaptation processes, like
differential imitation and cultural evolution in general
\citep[see][for various derivations of the
\rd]{Sandholm2013:Population-Game}. Under this interpretation, the \rd
is consistent with the idea that what is subject to the evolutionary
forces are not organisms but behavior: individuals can adapt their
behavior to the perceived environment or adopt strategies from other
individuals. Fitness captures the success of behavioral patterns,
which in the case of language can be thought of in terms of
communicative success.  Differential reproduction represents the
tendency of more successful strategies to be more likely adopted by
individuals, be it through imitation or some learning procedure within
the population.  

Under this non-biological interpretation, we have again two options of
picturing what diffusion is, similar to the biological cases
before. One possibility is that diffusion is noise in the adoption of
strategies, say, by conditional imitation, of strategies by other
signalers. Another possibility is that diffusion is again noise on the
realization of strategies: while behavior is optimized to be efficient
(be it due to learning, introspection or imitation), realization of
strategies is bound to be noisy due to confusability of similar
states. 

We believe that all of the four mentioned interpretations are, on
first approximations, feasible conceptualizations of the \rdd, and
that it is a good thing to know of a working account of vague
signaling that sketches where fitness-based selection under
state-confusability will take is, abstracting away from the details of
actually playing the game, inheriting, imitating or otherwise
optimizing behavior in whatever particular way. It is a good thing to
know this on the macro-level, especially since there are also
micro-level accounts that nicely complement the picture. We turn to
one such next.


\subsection{Relation with previous accounts}
\label{sec:relat-with-prev}

\paragraph{Generalized reinforcement.}
% Repeated below...
%\citet{OConnor2013:The-Evolution-o} introduced a generalization of
%Herrnstein reinforcement learning for sim-max games and showed that
%this not only leads to vague signaling patterns, but can speed-up
%evolution of efficient signaling strategies, especially in games with
%many states. 

Under plain Herrnstein reinforcement learning, sender and receiver
play the game repeatedly and adjust their dispositions to act after
each round of play, in such a way that the actual (non-negative)
payoff gained in the current interaction is added to the
non-normalized propensities for acting in exactly the way that they
acted in the current round of play. For sim-max games, this means that
when the sender chose $\messg$ in state $\state$, and this resulted in
some non-negative payoff (which is guaranteed by our choice of utility
function), the probability that the sender chooses $\messg$ again in
$\state$ is increased, but nothing else changes. In particular, the
senders behavior in other choice points does not change. Generalized
reinforcement learning is different here. When the use of $\messg$ in
$\state$ gave positive payoff, then not only will its future use
probability be promoted at $\state$ but also at other states,
proportional to how similar these are to $\state$. Similar amendments
take care of the way that the receiver updates his choice
dispositions.

\citet{OConnor2013:The-Evolution-o} shows that this extension not only
leads to vague signaling of the appropriate kind, but also speeds up
learning in such a way that, especially for games with higher numbers
of states, higher levels of communicative success are reached in
shorter learning periods than is possible without stimulus
generalization. Technically, this result is partly due to the fact
that signalers make bigger changes to their behavioral strategies
after each round of play under generalized reinforcement than under
the plain variety. But that only explains the speed of adaptation, not
necessarily that generalization also leads to regularity and
communicative efficiency. 

Diffusion of strategies in the \rdd can be conceived of as a form of
generalization as well, and works in large part quite analogous to
stimulus generalization in \citeauthor{OConnor2013:The-Evolution-o}'s
approach. This holds for the effect of diffusion and generalization,
but not necessarily for the way that the effect is achieved. We also
saw that diffusion in \rdd leads to more regular languages and
speedier convergence. However, the \rdd is a more abstract framework
than generalized reinforcement learning. The latter is foremost
motivated as a learning dynamic that has two players adapt their
individual strategies after each concrete round of play. In contrast,
the \rdd describes a more abstract, average dynamical change in
behavioral dispositions. Although the dynamics of (some forms of)
reinforcement learning mirror those of the replicator dynamics (at
some stage in time)
(\cite{BorgersSarin997:Learning-Throug,HopkinsPosch2005:Attainability-o,Beggs2005:On-the-Converge}),
this does not mean that \emph{generalized} reinforcement learning, in
which stimulus generalization is best motivated at the level of a
single agent's dispositional generalization after one round of play,
is also directly a plausible high-level description of, say,
generalized learning in a population of agents. Seen in this light,
generalized reinforcement learning and the \rdd nicely complement each
other, as similarly-minded accounts operating at different levels of
abstraction.

\paragraph{Quantal response equilibria.}
\citet{FrankeJager2010:Vagueness-Signa} suggested a number of ways in
which information processing limitations of signaling agents could
lead to vague strategies. The model that is most clearly related to
the present approach uses the notion of a quantal response, also known
as a logit response or a soft-max function
\citep[e.g.][]{Luce1959:Individual-Choi,McFadden1976:Quantal-Choice-,McKelveyPalfrey1995:Quantal-Respons,McKelveyPalfrey1998:Quantal-Respons,GoereeHolt2008:Quantal-Respons}. A
quantal response function is a paramterized generalization of the
classic best response function. For example, if $U \mycolon \Acts
\rightarrow \mathds{R}$ is the measure of expected utility over
choices $\Acts$ of an agent, then a best response function would have
the agent choose $\act$ only if $U(\act) = \max_{\act' \in \Acts}
U(a)$. A quantal response function rather assumes that agents would
choose $\act$ with a probability proportional to $\expo(\lambda \cdot
U(\act))$, where $\lambda$ is a rationality parameter. If $\lambda
\rightarrow \infty$ we retrieve the behavior of the best response
choice function, but if it is positive but finite, any choice $\act$
will receive a positive probability, but acts with higher expected
utility will be more likely. The underlying motivation for this choice
rule is the assumption that there is noise in the computation of
expected utilities and/or in maximization of expected
utilities. Consequently, choices with almost equal expected utilities
will be chosen with almost equal probability (for moderate values of
$\lambda$). 

\citet{FrankeJager2010:Vagueness-Signa} show that quantal response
equilibria of sim-max games, i.e., pairs of sender and receiver
strategies such that the sender strategy is the quantal response to
the expected utilities under the receiver strategy and vice versa, can
show the desired marks of vague
signaling. Figure~\ref{fig:exampleQRE_stratsA} gives an example of a
quantal response equilibrium for a sim-max game, as used in our set-up
but with higher tolerance $\toler = 0.5$. Sender and receiver
strategies look very much like what evolves under \rdd with modest
values of perceptual imprecision. 

\begin{figure}
  \centering
  
  \begin{subfigure}[]{0.45\textwidth}
    \includegraphics[width=\textwidth]{plots/exampleStratQRE_tolerance05.pdf}
    \caption{$\ns = 90$, $\lambda = 15$, $\toler = 0.5$}
    \label{fig:exampleQRE_stratsA}
  \end{subfigure}
  \hfill
  \begin{subfigure}[]{0.45\textwidth}
    \includegraphics[width=\textwidth]{plots/exampleStratQRE_tolerance005.pdf}
    \caption{$\ns = 90$, $\lambda = 15$, $\toler = 0.05$}
    \label{fig:exampleQRE_stratsB}
  \end{subfigure}

  \caption{Examples of vague quantal response equilibria.}
  \label{fig:exampleQREs}
\end{figure}


But not all quantal response equilibria are equally plausible
explanations of vague language use, and the ones that are not suggest
that it is less plausible to think of vagueness as arising because of
errors in the computation and maximization of expected utility, as the
quantal response approach assumes, than that it arises due to
confusion of similar states, as the \rdd proposes. To see what the
problem is, we can look at cases like given in
Figure~\ref{fig:exampleQRE_stratsB}, which is the quantal response
equilibrium for a game with lower tolerance $\toler = 0.05$. Unlike
what evolves under \rdd in this case, sender strategies have vague
boundaries also towards the end of the unit intervals. Technically,
this is because quantal responses equalize message use far away from
the ``prototypical'' interpretation, not just in-between categories,
so to speak. This, in turn, is because quantal responses introduce
noise into the decision making at the level of computing or maximizing
expected utility of choices.

That this is conceptually odd shows even more clearly in a case where
the state space is intuitively unbounded, as for instance for the
property ``tall''. If the usual interpretation of a ``tall man'' peaks
at around, say, 195cm then when meeting a giant of $n$ meters senders
would, according to the quantal response approach, be ever more
inclined to describe the giant indifferently as either ``tall'' or
``short'' the larger $n$ gets. This is because, as the distance from
the prototype increases for larger $n$, the difference between the
expected utilities of saying ``short'' or ``tall'' will converge to
zero. 
%Whence that the quantal response approach would predict that
%senders would grow indifferent between choice of antonyms as $n$
%grows, which seems weird. 

Admittedly, this argument hinges on the choice of utility
function. Still, to the extent that the chosen lower bounded utility
functions are reasonable ---and we think they are very reasonable---,
the case suggests that quantal responses are not a good model,
intuitively speaking, for why linguistic categories are
vague. Vagueness is more plausibly an effect of perceptual confusion
of similar states, than of computational errors in maximizing expected
utility.

%%% Local Variables: 
%%% mode: latex
%%% TeX-master: "paper"
%%% TeX-PDF-mode: t
%%% End:





\section{Conclusions}
\label{sec:conclusions}

%%% Local Variables: 
%%% mode: latex
%%% TeX-master: "paper"
%%% TeX-PDF-mode: t
%%% End:





\appendix

\section{Proof of Theorem}
\label{sec:proofs}

\begin{proof}[Part (i).]
  Fix $\Smixed$ and $\Sstrat = G(\Smixed)$. Look first at the rhs of
  the consequent:
  \begin{align*}
    \Diff_C(\Sstrat)(\messg_y \probbar \state_x) & =  \sum_{\state_l} C_{xl}
    \cdot \Sstrat(\messg_y \probbar \state_l) && \text{(by Equation~(\ref{eq:confusion-function}))} \\
    & =  \sum_{\state_l} C_{xl}
    \cdot  \sum_{\Spure_i(\state_l) = \messg_y} \Smixed_i && \text{(by Equation~(\ref{eq:CorrespondenceBehavioralMixed}))} \\
    & = \sum_{\state_l}
    \sum_{\Spure_i(\state_l) = \messg_y} \Smixed_i \cdot C_{xl}\,.
  \end{align*}
  Next consider the lhs of the consequent:
  \begin{align*}
    G(\Mutate_{Q^C}(\Smixed))(\messg_y \probbar \state_x) & =
    \sum_{\Spure_i(\state_x)=\messg_y} \Mutate_{Q^C}(\Smixed_i) &&
    \text{(by Equation~(\ref{eq:CorrespondenceBehavioralMixed}))} \\
    & = \sum_{\Spure_i(\state_x)=\messg_y} \sum_{\Spure_j}
    \Smixed_j \cdot Q^C_{ji} &&
    \text{(by Equation~(\ref{eq:Mutation}))} \\
    & = \sum_{\Spure_i(\state_x)=\messg_y} \sum_{\Spure_j}
    \Smixed_j \cdot \prod_{\state_l} \sum_{\state_m \in
      \Spure_j^{-1}(\Spure_i(\state_l))} C_{lm} &&
    \text{(by Equation~(\ref{eq:construction-sen}))} \\
    & = \sum_{\Spure_j} \Smixed_j \cdot
    \sum_{\Spure_i(\state_x)=\messg_y} \prod_{\state_l}
    \sum_{\state_m \in \Spure_j^{-1}(\Spure_i(\state_l))} C_{lm}
  \end{align*}
  To simplify this further we look at a fixed $\Spure_j$ and consider
  the term: 
  \begin{align}
    \label{eq:term}
    \sum_{\Spure_i(\state_x)=\messg_y} \prod_{\state_l} \sum_{\state_m
      \in \Spure_j^{-1}(\Spure_i(\state_l))} C_{lm}\,.
  \end{align}
  Let $Y$ be the row-stochastic matrix with $Y_{kl} = \sum_{\state_m
    \in \Spure_j^{-1}(\messg_l)} C_{km}$. Every pure sender strategy
  maps each state $\state_k$ onto exactly one $Y_{kl}$. If we quantify
  over all pure strategies, we essentially look at each such
  mapping. Term (\ref{eq:term}) above sums over all pure strategies
  that map $\state_k$ onto $\messg_y$. The above term then sums over
  all products whose factors are tuples in $\times_{k>2} \set{y
    \setbar \exists l \mycolon y = Y_{kl}}$. So term (\ref{eq:term})
  expands to (where $e = \card{\States}$ and $d=\card{\Messgs}$):
  \begin{align*}
    & (Y_{11} \cdot Y_{21} \cdot Y_{31} \cdot \ldots \cdot Y_{e1}) +
    (Y_{11} \cdot Y_{21} \cdot Y_{31} \cdot \ldots \cdot Y_{e2}) + 
    \dots \\
    & + (Y_{11} \cdot Y_{2d} \cdot
    Y_{3d} \cdot \ldots \cdot
    Y_{ed}) 
  \end{align*}
  But since $Y$ is a row-stochastic matrix, this simplifies to
  $Y_{xy}$. Continuing the derivation with this:
  \begin{align*}
    G(\Mutate_{Q^C}(\Smixed))(\messg_y \probbar \state_x) 
    & = \sum_{\Spure_j} \Smixed_j \cdot
    \sum_{\state_l \in \Spure_j^{-1}(\messg_y)} C_{xl} \\
    & = \sum_{\Spure_j} \sum_{\state_l \in \Spure_j^{-1}(\messg_y)} \Smixed_j \cdot
     C_{xl} \\
     & = \sum_{\state_l}
    \sum_{\Spure_i(\state_l) = \messg_y} \Smixed_i \cdot C_{xl}\,.
  \end{align*}

\end{proof}

\begin{proof}[Part (ii).]
  Fix $\Rmixed$ and $\Rstrat = G(\Rmixed)$. The rhs of the consequent
  expands to:
  \begin{align*}
    \Diff_C(\Rstrat)(\state_x \probbar \messg_y) & = \sum_{\state_j}
    \sum_{i \in \set{k \setbar \Rpure_k(\messg_y) = \state_x}}
    \Rmixed_i \cdot C_{jx} && \text{(by Equation~(\ref{eq:confusion-function}))} \\
    & = \sum_{\Rpure_i} \sum_{j \in \set{k \setbar \Rpure_i(\messg_y) = \state_j}}
    \Rmixed_i \cdot C_{jx} \\
    & = \sum_{\Rpure_i} \Rmixed_i \cdot C_{\Rpure_i(\messg_y)x} \,.
  \end{align*}
  The rhs expands to (by Equations~(\ref{eq:CorrespondenceBehavioralMixed}),
  (\ref{eq:Mutation}) and (\ref{eq:construction-rec})):
  \begin{align*}
    G(\Mutate_{R^C}(\Rmixed))(\state_x \probbar \messg_y) & = \sum_{\Rpure_i}
    \Rmixed_i \cdot \sum_{j \in \set{j \setbar \Rpure_k(\messg_y) =
        \state_x}} 
    \prod_{\messg} C_{\Rpure_i(\messg)\Rpure_j(\messg)} \,.
  \end{align*}
  Without loss of generality, assume that $x=y=1$, and fix
  $\card{M}=d$ and let $e$ be the number of pure receiver
  strategies. Then the last term can be rewritten as:
  \begin{align*}
    & = \sum_{i}
    \Rmixed_i \cdot ( C_{\Rpure_i(\messg_1)1} \cdot
      C_{\Rpure_i(\messg_2)\Rpure_1(\messg_2)} \cdot \ldots \cdot
      C_{\Rpure_i(\messg_d)\Rpure_1(\messg_d)} + \ldots  \\
      & \textcolor{white}{ = \sum_{i}
    \Rmixed_i  (}  +  C_{\Rpure_i(\messg_1)1} \cdot
      C_{\Rpure_i(\messg_2)\Rpure_2(\messg_2)} \cdot \ldots \cdot
      C_{\Rpure_i(\messg_d)\Rpure_2(\messg_d)} + \ldots  \\
      & \textcolor{white}{ = \sum_{i}
    \Rmixed_i  (}  + C_{\Rpure_i(\messg_1)1} \cdot
      C_{\Rpure_i(\messg_2)\Rpure_e(\messg_2)} \cdot \ldots \cdot
      C_{\Rpure_i(\messg_d)\Rpure_e(\messg_d)}) ) \,.
  \end{align*}
  For every messages $\messg_l$, $C_{\Rpure_i(\messg_l)}$ is a stochastic
  vector. For $l>1$, all elements of these vectors appear equally
  often. But that means that these cancel out. What remains is:
  \begin{align*}
    & = \sum_{\Rpure_i} \Rmixed_i \cdot C_{\Rpure_i(\messg_y)x} \,.
  \end{align*}
\end{proof}


%%% Local Variables: 
%%% mode: latex
%%% TeX-master: "paper"
%%% TeX-PDF-mode: t
%%% End:






\printbibliography[heading=bibintoc]

\end{document}
