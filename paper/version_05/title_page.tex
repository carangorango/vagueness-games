\documentclass{article}
\usepackage[utf8]{inputenc}



\title{Vagueness and imprecise imitation in signaling games}

\author{
Michael Franke%
\thanks{Department of Linguistics, Faculty of Humanities, University of Tübingen, Germany}\\
\texttt{mchfranke@gmail.com}
\and
Jos\'e Pedro Correia%
\thanks{Department of Philosophy, University of Porto, Portugal}\\
\texttt{zepedro.correia@gmail.com}
}

% \usepackage{a4wide}

\date{}

\begin{document}

\noindent \textbf{Title:} 

\medskip

\noindent Vagueness and imprecise imitation in signaling games

\medskip
\medskip
\medskip

\noindent \textbf{Authors:}

\medskip

\begin{minipage}{0.4\linewidth}
  Michael Franke \\
  Wilhelmstra\ss e 19\\
  Department of Linguistics\\
  University of T\"ubingen\\
  72072 T\"u bingen \\
  Germany \\
\end{minipage}
\hfill
\begin{minipage}{0.5\linewidth}
  Jos\'e Pedro Correia\\
  Praça Rainha D. Amélia 268 4 HAB3\\
  4000-075 Porto\\
  Portugal \\
\end{minipage}

\medskip
\medskip
\medskip

\noindent \textbf{Abstract:}

\medskip 

\noindent  Signaling games are popular models for studying the evolution of meaning, but typical
  approaches do not incorporate vagueness as a feature of successful signaling.  Complementing
  recent like-minded models, we describe an aggregate population-level dynamic that describes a
  process of imitation of successful behavior under imprecise perception and realization of
  similar stimuli. Applying this new dynamic to a generalization of Lewis' signaling games, we
  show that stochastic imprecision leads to vague, yet by-and-large efficient signal use, and,
  moreover, that it unifies evolutionary outcomes and helps avoid suboptimal
  categorization. The upshot of this is that we see \emph{as-if}-generalization at an aggregate
  level, without agents actually generalizing.

\medskip
\medskip
\medskip

\noindent \textbf{Acknowledgements:}

\medskip 

\noindent We would like to thank Gerhard J\"ager, Robert van Rooij and Martin Stokhof for
inspiring discussion and comments. MF gratefully acknowledges financial support during the
completion of this work by NWO-VENI grant 275-80-004, the Institutional Strategy of the
University of T\"ubingen (Deutsche Forschungsgemeinschaft, ZUK 63) and the Priority Program
XPrag.de (DFG Schwerpunktprogramm 1727).

\end{document}