\appendix

\section{Proof of Theorem}
\label{sec:proofs}

\begin{proof}[Part (i).]
  Fix $\Smixed$ and $\Sstrat = G(\Smixed)$. Look first at the rhs of
  the consequent:
  \begin{align*}
    C(\Sstrat)(\messg_y \probbar \state_x) & =  \sum_{\state_l} C_{xl}
    \cdot \Sstrat(\messg_y \probbar \state_l) \\
    & =  \sum_{\state_l} C_{xl}
    \cdot  \sum_{\Spure(\state_l) = \messg_y} \Smixed(\Spure) \\
    & = \sum_{\state_l}
    \sum_{\Spure(\state_l) = \messg_y} \Smixed(\Spure) \cdot C_{xl}\,.
  \end{align*}
  Next consider the lhs of the consequent:
  \begin{align*}
    G(Q^C(\Smixed))(\messg_y \probbar \state_x) & =
    \sum_{\Spure_i(\state_x)=\messg_y} Q^C(\Smixed)(\Spure_i) \\
    & = \sum_{\Spure_i(\state_x)=\messg_y} \sum_{\Spure_j}
    \Smixed(\Spure_j) \cdot Q^C_{ji} \\
    & = \sum_{\Spure_i(\state_x)=\messg_y} \sum_{\Spure_j}
    \Smixed(\Spure_j) \cdot \prod_{\state_l} \sum_{\state_m \in
      \Spure_j^{-1}(\Spure_i(\state_l))} C_{lm} \\
    & = \sum_{\Spure_j} \Smixed(\Spure_j) \cdot
    \sum_{\Spure_i(\state_x)=\messg_y} \prod_{\state_l}
    \sum_{\state_m \in \Spure_j^{-1}(\Spure_i(\state_l))} C_{lm}
  \end{align*}
  To simplify this further we look at a fixed $\Spure_j$ and consider
  the term: 
  \begin{align}
    \label{eq:term}
    \sum_{\Spure_i(\state_x)=\messg_y} \prod_{\state_l} \sum_{\state_m
      \in \Spure_j^{-1}(\Spure_i(\state_l))} C_{lm}\,.
  \end{align}
  Let $Y$ be the row-stochastic matrix with $Y_{kl} = \sum_{\state_m
    \in \Spure_j^{-1}(\messg_l)} C_{km}$. Every pure sender strategy
  maps each state $\state_k$ onto exactly one $Y_{kl}$. If we quantify
  over all pure strategies, we essentially look at each such
  mapping. Term (\ref{eq:term}) above quantifies over all pure
  strategies that map $\state_k$ onto $\messg_y$. The above term then
  sums over all products whose factors are tuples in $\times_{k>2}
  \set{y \setbar \exists l \mycolon y = Y_{kl}}$. So term
  (\ref{eq:term}) expands to (where $e = \card{\States}$ and $d=\card{\Messgs}$):
  \begin{align*}
    & (Y_{11} \cdot Y_{21} \cdot Y_{31} \cdot \ldots \cdot Y_{e1}) +
    (Y_{11} \cdot Y_{21} \cdot Y_{31} \cdot \ldots \cdot Y_{e2}) + 
    \dots \\
    & + (Y_{11} \cdot Y_{2d} \cdot
    Y_{3d} \cdot \ldots \cdot
    Y_{ed})\\
    = &    
  \end{align*}
  But since $Y$ is a row-stochastic matrix, this simplifies to
  $Y_{xy}$. Continuing the derivation with this:
  \begin{align*}
    G(Q^C(\Smixed))(\messg_y \probbar \state_x) 
    & = \sum_{\Spure_j} \Smixed(\Spure_j) \cdot
    \sum_{\state_l \in \Spure_j^{-1}(\messg_y)} C_{xl} \\
    & = \sum_{\Spure_j} \sum_{\state_l \in \Spure_j^{-1}(\messg_y)} \Smixed(\Spure_j) \cdot
     C_{xl} \\
     & = \sum_{\state_l}
    \sum_{\Spure(\state_l) = \messg_y} \Smixed(\Spure) \cdot C_{xl}\,.
  \end{align*}

\end{proof}

\begin{proof}[Part (ii).]
  Fix $\Rmixed$ and $\Rstrat = G(\Rmixed)$. The rhs of the consequent
  expands to:
  \begin{align*}
    C(\Rstrat)(\state_x \probbar \messg_y) & = \sum_{\state_j}
    \sum_{i \in \set{k \setbar \Rpure_k(\messg_y) = \state_x}}
    \Rmixed_i \cdot C_{jx} \\
    & = \sum_{\Rpure_i} \sum_{j \in \set{k \setbar \Rpure_i(\messg_y) = \state_j}}
    \Rmixed_i \cdot C_{jx} \\
    & = \sum_{\Rpure_i} \Rmixed_i \cdot C_{\Rpure_i(\messg_y)x} \,.
  \end{align*}
  The rhs expands to:
  \begin{align*}
    G(R^C(\Rmixed))(\state_x \probbar \messg_y) & = \sum_{\Rpure_i}
    \Rmixed_i \cdot \sum_{j \in \set{j \setbar \Rpure_k(\messg_y) =
        \state_x}} 
    \prod_{\messg} C_{\Rpure_i(\messg)\Rpure_j(\messg)}
  \end{align*}
  Without loss of generality, assume that $x=y=1$, and fix
  $\card{M}=d$ and let $e$ be the number of pure receiver
  strategies. Then the last term can be rewritten as:
  \begin{align*}
    & = \sum_{i}
    \Rmixed_i \cdot ( C_{\Rpure_i(\messg_1)1} \cdot
      C_{\Rpure_i(\messg_2)\Rpure_1(\messg_2)} \cdot \ldots \cdot
      C_{\Rpure_i(\messg_d)\Rpure_1(\messg_d)} + \ldots  \\
      & \textcolor{white}{ = \sum_{i}
    \Rmixed_i  (}  +  C_{\Rpure_i(\messg_1)1} \cdot
      C_{\Rpure_i(\messg_2)\Rpure_2(\messg_2)} \cdot \ldots \cdot
      C_{\Rpure_i(\messg_d)\Rpure_2(\messg_d)} + \ldots  \\
      & \textcolor{white}{ = \sum_{i}
    \Rmixed_i  (}  + C_{\Rpure_i(\messg_1)1} \cdot
      C_{\Rpure_i(\messg_2)\Rpure_e(\messg_2)} \cdot \ldots \cdot
      C_{\Rpure_i(\messg_d)\Rpure_e(\messg_d)}) )
  \end{align*}
  Every for every messages $C_{\Rpure_i(\messg_l)}$ is a stochastic
  vector. For $l>1$, all elements of these vectors appear equally
  often. But that means that these cancel out. What remains is:
  \begin{align*}
    & = \sum_{\Rpure_i} \Rmixed_i \cdot C_{\Rpure_i(\messg_y)x} \,.
  \end{align*}
\end{proof}


%%% Local Variables: 
%%% mode: latex
%%% TeX-master: "paper"
%%% TeX-PDF-mode: t
%%% End:



