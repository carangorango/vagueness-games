\section{Discussion}
\label{sec:discussion}

\subsection{Model interpretation}
The replicator-mutator equations are a mathematical model motivated by concepts from the theory of natural selection, namely natural variation, differential reproduction, and mutation.
The most straightforward interpretation is thus a biological one: we can imagine signaling strategies as innate behaviorial tendencies of organisms, steps in the evolutionary process as successive generations, and selection as capturing the reproductive advantage of fitter individuals.
In this interpretation, our proposed model is not an extremely interesting one, since the noise matrix simply corresponds to a restricted form of mutation.

Another possibility is to take the equations as embodying cultural evolution.
In this interpretation, what is subject to the evolutionary forces are not organisms but strategies themselves.
This pressuposes that individuals can adopt strategies from other individuals.
Fitness captures the success of strategies, which in the case of language can be thought of in terms of communicative success.
Differential reproduction represents the tendency of more successful strategies to be more likely adopted by individuals, be it through imitation or some learning procedure within the population.
Under this interpretation, the proposed noise perturbation becomes more interesting.
Messages correspond to the overt data available to individuals, whereas states are interpreted as the more private aspect of the behavior.
Futhermore, similarity-maximization games are especially interesting in situations where the number of states largely exceeds the number of messages available.
When trying to adopt another individual's behavior, it is thus more likely that information about states is both less accessible and less practical to elicit, hence justifying a higher likelihood of mistakes and a more pressing need for a certain degree of generalization%
\footnote{The consequence of perturbing a strategy with a noise matrix based on similarity can be seen as ``spreading'' the effects of the fitness of the behavior for a state to its more similar states, which can also be interpreted as generalization}%
.

Finally, the replicator dynamics are also related to some forms of reinforcement learning (\cite{Borgers1997,Hopkins2005,Beggs2005}), and can thus be interpreted in cognitive terms.



\mytodo{JPC}{Discuss nature of noise}
\mytodo{JPC}{Compare with O'Connor}
\mytodo{JPC}{Compare with Franke2013}

%%% Local Variables: 
%%% mode: latex
%%% TeX-master: "paper"
%%% TeX-PDF-mode: t
%%% End:



