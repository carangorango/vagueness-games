\documentclass{article}
\usepackage[utf8]{inputenc}

\title{Vagueness and \emph{as-if}-generalization from imprecise imitation in signaling games}

\author{
Michael Franke%
\thanks{Department of General Linguistics, Faculty of Humanities, University of Tübingen}\\
\texttt{mchfranke@gmail.com}
\and
Jos\'e Pedro Correia\\
\texttt{zepedro.correia@gmail.com}
}

\date{}

\begin{document}

\maketitle

\begin{abstract}
  Signaling games are popular models for studying the evolution of meaning, but typical
  approaches do not incorporate vagueness as a feature of successful signaling.  Complementing
  recent like-minded models, we describe an aggregate population-level dynamic that describes a
  process of imitation of successful behavior under imprecise perception and realization of
  similar stimuli. Applying this new dynamic to a generalization of Lewis' signaling games, we
  show that stochastic imprecision leads to vague, yet by-and-large efficient signal use, and,
  moreover, that it unifies evolutionary outcomes and helps avoid suboptimal
  categorization. The upshot of this is that we see \emph{as-if}-generalization at an aggregate
  level, without agents actually generalizing.
\end{abstract}

\end{document}