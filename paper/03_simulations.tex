\section{Simulations}

In order to study the long-term behavior of the noise-perturbed sim-max games, we conducted computer simulations for different combinations of initial parameters.
Namely, we were interested in investigating the impact of different values of impairment and its relation with the size of the state space.
\mytodo{JPC}{Justify why size of state space is interesting}

\subsection{Measures}
To be able to quantitively analyze the results of our simulations, we define a number of measures to capture different characteristics of sim-max languages.

\paragraph{Entropy.}
This classic information-theoretic measure captures the amount of uncertainty in a signaling strategy.
The most natural way to define it is in terms of mixed strategies rather than behavioral strategies.
Let us recall that mixed sender (receiver) strategies are functions $\Smixed \in \Delta(\Messgs^\States)$ ($\Rmixed \in (\Acts^\Messgs)$).
We can define the entropy of a sender strategy as
\begin{align*}
  E(\Smixed) = \sum_{\Spure \in \Messgs^\States} \Smixed(\Spure) \cdot \log(\Smixed(\Spure))
\end{align*} 
and the entropy of a receiver strategy as
\begin{align*}
  E(\Rmixed) = \sum_{\Rpure \in \Acts^\Messgs} \Rmixed(\Rpure) \cdot \log(\Rmixed(\Rpure)) \,.
\end{align*} 
These measures are computationally expensive to calculate, since the size of the domain over which the sum is computed grows exponentially with the number of choice points.
Therefore, we converted\footnote{See Appendix~\ref{sec:proofs}.} the above definitions into equivalent measures defined in terms of behavioral strategies.
These are as follows:
\begin{align*}
  E(\Sstrat) = -\sum_{\state \in \States} \sum_{\messg \in \Messgs} \Sstrat(\messg \probbar \state) \cdot \log(\Sstrat(\messg \probbar \state))
\end{align*} 
\begin{align*}
  E(\Rstrat) = -\sum_{\messg \in \Messgs} \sum_{\act \in \Acts} \Rstrat(\act \probbar \messg) \cdot \log(\Rstrat(\act \probbar \messg)) \,.
\end{align*}
The measures are lower bounded by $0$ and upper bounded by, respectively, $\log(|\Messgs^\States|) = |\States| \cdot \log(|\Messgs|)$ and $\log(|\Acts^\Messgs|) = |\Messgs| \cdot \log(|\Acts|)$, thus we can normalize them by dividing by these values.

\paragraph{Informativity.}
\mytodo{JPC}{Informativity is perhaps a bad name for the receiver...}
The quantity of information in a signal is an old notion that also goes back to the start of information theory.
Skyrms~\citet[ch.~3]{Skyrms2010} discusses its use in the context of signaling games.
The main idea is that we can quantify the amount of information about a state $\state$ in a message $\messg$ via the relation between the probability of the state given the message $\Pr(\state \,|\, \messg)$ and the unconditional probability of the state $\Pr(\state)$.
Following Bayes' theorem, we can express $\Pr(\state \,|\, \messg)$ as $\frac{\Pr(\state) \cdot \Pr(\messg \,|\, \state)}{\Pr(\messg)}$.
We then have $\Pr(\messg \,|\, \state) = \Sstrat(\state,\messg)$ and $\Pr(\messg) = \sum_{\state^\prime \in \States} \Pr(\state^\prime) \cdot \Sstrat(\state^\prime,\messg)$.
Finally, we equate sender infomativity with the average overall information about states in each signal.
Based on the definition by Skyrms~\citet[p.~36]{Skyrms2010}, and the considerations above, we define sender informativity as follows:
\begin{align*}
  I(\Sstrat) = \frac{1}{|\Messgs|} \sum_{\messg \in \Messgs} \sum_{\state \in \States} \frac{\Pr(\state) \cdot \Sstrat(\state,\messg)}{\sum_{\state^\prime \in \States} \Pr(\state^\prime) \cdot \Sstrat(\state^\prime,\messg)} \cdot \log \left(\frac{\Sstrat(\state,\messg)}{\sum_{\state^\prime \in \States} \Pr(\state^\prime) \cdot \Sstrat(\state^\prime,\messg)} \right) \,.
\end{align*}
Conversely, we can quantify the amount of information about an act in a message.
We equate receiver informativity with the average overall information about acts in each signal.
Based on Skyrms' definition~\citet[p.~39]{Skyrms2010}, we define receiver informativity as follows:
\begin{align*}
  I(\Sstrat,\Rstrat) = \frac{1}{|\Messgs|} \sum_{\messg \in \Messgs} \sum_{\act \in \Acts} \Rstrat(\messg,\act) \cdot \log \left(\frac{\Rstrat(\messg,\act)}{\sum_{\state \in \States} \Pr(\state) \cdot \sum_{\messg^\prime \in \Messgs} \Sstrat(\state,\messg^\prime) \cdot \Rstrat(\messg^\prime,\act)} \right) \,.
\end{align*}
Both measures are bounded between $0$ and $1$.

\paragraph{Voronoiness.}
This measure aims to quantify how close a strategy is to being a part of a Voronoi tessallation of the state space.
Recalling the results by \cite{Jager2007}, the stable outcomes of a population of agents playing a similarity-maximization game using pure strategies and evolving according to the replicator dynamics are those where ``the sender strategy is consistent with the Voronoi tessallation that is induced by the image of the receiver strategy.''~(\cite{Jager2007}, p. 562).
\mytodo{JPC}{Explain further?}
Given we are working with behavioral strategies, rather than pure strategies, and that we perform numerical simulations, rather than calculate analytical results, instead of looking into a binary criterion of whether a simulation result is such a Voronoi tessallation of the state space or not, we want a more progressive measure to characterize that in terms of degrees.
With that in mind, and given that this measure only makes sense for $\Acts = \States$, we define Voronoiness as follows.
Let $\pi(\Rstrat, \messg) = \argmax_{\state \in \States}\Rstrat(\messg,\state)$ be the prototype of a message $\messg \in \Messgs$, then
\begin{multline*}
  V(\Sstrat, \Rstrat) = \sum_{\state \in \States} \sum_{\messg_1 \in \Messgs} \sum_{\messg_2 \in \Messgs} \textrm{if } s(\state, \pi(\Rstrat, \messg_1)) > s(\state, \pi(\Rstrat, \messg_2)) \bicond \Sstrat(\state, \messg_1) > \Sstrat(\state, \messg_2) \textrm{ then } 1 \textrm{ else } 0
\end{multline*}
is the Voronoiness of the sender strategy $\Sstrat$ given the receiver strategy $\Rstrat$ and
\begin{align*}
  V(\Rstrat) = \sum_{\messg \in \Messgs} \sum_{\state_1 \in \States} \sum_{\state_2 \in \States} \textrm{if } s(\state_1, \pi(\Rstrat, \messg)) > s(\state_2, \pi(\Rstrat, \messg)) \bicond \Rstrat(\messg, \state_1) > \Rstrat(\messg, \state_2) \textrm{ then } 1 \textrm{ else } 0
\end{align*}
is the Voronoiness of the receiver strategy $\Rstrat$.
\mytodo{JPC}{Give some intuitions on these?}
The measures are lower bounded by $0$ and upper bounded by, respectively, $|\States| \times |\Messgs|^2$ and $|\Messgs| \times |\States|^2$, thus we can normalize them by dividing by these values.

\subsection{Methodology}
\newcommand{\impairment}{\alpha}
In our simulations we focus on a concrete setup in line with \cite{Correia2013}.
Namely, we define a similarity-maximization game with $\States = \Acts$, where $\States$ is a discrete set of equidistant points within the unit interval $[0,1]$, including the boundaries (thus we always use $n \geq 2$).
As similarity measure $s:\States \times \States \rightarrow [0,1]$, we use a negative exponential of the of the square distance, \emph{i.e.}~for $\state,\state^\prime \in \States$:
\begin{align*}
  s(\state,\state^\prime) =
    \begin{cases}
    1 & \textrm{if } \impairment = 0 \textrm{ and } \state = \state^\prime \\
    0 & \textrm{if } \impairment = 0 \textrm{ and } \state \neq \state^\prime \\
    e^{-\frac{\lVert \state - \state^\prime \rVert ^2}{\impairment^2}} & \textrm{otherwise} \\
    \end{cases}
\end{align*}
where $\impairment$ is an \emph{impairment} factor.
Here we deviate from \cite{Correia2013} by working with impairment $\impairment$ rather than acuity $\theta$, since we are interested in exploring the case of having $0$ impairment, and acuity does not have a natural maximum as defined.
Confusion is directly related with similarity, \emph{i.e.}~the more similar two states are, the more likely an agent is to perceive one as the other.
For simplicity, we equate one with the other by defining $C_{ij} = s(\state_i,\state_j)$.

We run the noise-perturbed replicator dynamics with random initial strategies $\Sstrat_0$ and $\Rstrat_0$ until a convergence criterion is met, namely that the total absolute change in both sender and receiver strategies does not amount to more than $0.01$, \emph{i.e.}:
\begin{align*}
  \sum_{\state \in \States} \sum_{\messg \in \Messgs} \left| \Sstrat^\prime(\state,\messg) - \Sstrat(\state,\messg) \right| < 0.01
\end{align*}
and
\begin{align*}
  \sum_{\messg \in \Messgs} \sum_{\state \in \States} \left| \Rstrat^\prime(\messg,\state) - \Rstrat(\messg,\state) \right| < 0.01 \,.
\end{align*}

We perform a parameter sweep with size of the state space $n \in \{10, 50, 100, 200\}$ and impairment $\impairment \in \{0, 0.05, 0.1, 0.15, 0.2, 0.25, 0.3, 0.35, 0.4\}$.
For each parameter combination, we perform $100$ trials.

%%% Local Variables: 
%%% mode: latex
%%% TeX-master: "paper"
%%% TeX-PDF-mode: t
%%% End:



