\section{Simulations}

In order to study the long-term behavior of the noise-perturbed sim-max games, we conducted computer simulations for different combinations of initial parameters.
Namely, we were interested in investigating the impact of different values of impairment and its relation with the size of the state space.
\mytodo{JPC}{Justify why size of state space is interesting}

\subsection{Measures}
To be able to quantitively analyze the results of our simulations, we decided to define a number of measures to capture different characteristics of sim-max languages.

\paragraph{Entropy.}
This classic information-theoretic measure captures the amount of uncertainty in a signaling strategy.
The most natural way is to define it in terms of mixed strategies rather than behavioral strategies.
Let us recalle that mixed sender (receiver) strategies are functions $\Smixed \in \Delta(\Messgs^\States)$ ($\Rmixed \in (\Acts^\Messgs)$).
We can define the entropy of a sender strategy (henceforth $\Sstrat$-entropy) as:
\begin{align*}
  E(\Smixed) = \sum_{\Spure \in \Messgs^\States} \Smixed(\Spure) \cdot \ln(\Smixed(\Spure)) \,.
\end{align*} 
and the entropy of a receiver strategy (henceforth $\Rstrat$-entropy) as:
\begin{align*}
  E(\Rmixed) = \sum_{\Rpure \in \Acts^\Messgs} \Rmixed(\Rpure) \cdot \ln(\Rmixed(\Rpure)) \,.
\end{align*} 
These measures are computationally expensive to calculate, since the size of the domain over which the sum is computed grows exponentially with the number of choice points.
Therefore, we converted\footnote{See Appendix~\ref{sec:proofs}.} the above definitions into equivalent measures defined in terms of behavioral strategies.
These are as follows:
\begin{align*}
  E(\Sstrat) = -\sum_{\state \in \States} \sum_{\messg \in \Messgs} \Sstrat(\messg \probbar \state) \cdot \ln(\Sstrat(\messg \probbar \state)) \,.
\end{align*} 
\begin{align*}
  E(\Rstrat) = -\sum_{\messg \in \Messgs} \sum_{\act \in \Acts} \Rstrat(\act \probbar \messg) \cdot \ln(\Rstrat(\act \probbar \messg)) \,.
\end{align*}
The measures are lower bounded by $0$ and upper bounded by, respectively, $\ln(|\Messgs^\States|) = |\States| \cdot ln(|\Messgs|)$ and $\ln(|\Acts^\Messgs|) = |\Messgs| \cdot ln(|\Acts|)$, thus we can normalize them by dividing by these values.

%%% Local Variables: 
%%% mode: latex
%%% TeX-master: "paper"
%%% TeX-PDF-mode: t
%%% End:



