\section{Introduction}

Many of our concepts and words are vague. A vague category knows clear
cases that fall under it, clear cases that do not, and also so-called
borderline cases in-between that neither clearly apply, nor clearly
not apply, but might themselves not be equal in terms of how well they
represent the category in question. Vagueness does not seem to
dramatically affect the success of everyday communication, but it is
troublesome for some of our best theories of language. Especially the
logico-semantic tradition of Frege, Russell and the young Wittgenstein
is challenged by vagueness and the paradoxes that it gives rise to. As
can be expected, many proponents of this tradition have risen to the
challenge
\citep[e.g.][]{Williamson1994:Vagueness,CobrerosEgre2012:Tolerant-Classi}.\todo{more
  references}

Yet there are other intriguing aspects about vagueness. The
puzzle that we are concerned with here is how vagueness could arise
and be maintained in the first place. This may not seem a particularly
deep issue at first glance, but on closer inspection it is a serious
worry to functionalist accounts that maintain that our concepts and
language use evolved in order to be efficient. Since the existence of
unclear borderline cases seems to entail inefficiency of
categorization or communication, the challenge, succinctly put forward
by \citet{Lipman2009:Why-is-Language}, is to explain how vagueness can
exist despite its obvious and demonstrable functional deficiency under
evolutionary pressure to be optimally expressive.

This problem, call it Lipman's problem, has a conceptual and a
technical side to it. Conceptually, it asks for reasons and general
mechanisms by which we could plausibly conceive of vagueness as
resisting the pressure of evolutionary selection for precision. Such
reasons and general mechanisms are, arguably, not hard to imagine. We
will address some presently. But each putative explanation, no matter
how plausible intuitively, must also be checked for internal
consistency and its potential to account for the fact that we see
vagueness as a part of a by-and-large efficient system of
categorization and communication. In other words, not all first-shot
rebuttals of Lipman's problem will do. Denying that there is any
functional pressure on efficient communication, for instance, is a
non-starter, because it leaves the relative efficiency of our
communication with vague words entirely unexplained. In sum, the
technical answer to Lipman's problem is to give, ideally, a model of
evolution of categorization or language use that shows how competing
forces that realize both by-and-large efficient categories and
vagueness.

A number of authors have recently tried to explain why vagueness
evolved, based on considerations why a vague language might be useful
\citep[e.g.][]{Jaegherde-Jaegher2003:A-Game-Theoreti,Deemter2009:Utility-and-Lan,Jaegherde-JaegherRooijvan-Rooij2010:Strategic-Vague,BlumeBoard2013:Intentional-Vag}. In
contrast to these, others have argued that vagueness is a natural
byproduct of limitations in information processing
\citep[e.g.][]{FrankeJager2010:Vagueness-Signa,OConnor2013:The-Evolution-o,QingFranke2014:Gradable-Adject}. This
paper makes a contribution to the latter line of thought.  Concretely,
we introduce a novel variant of the replicator mutator dynamic that
integrates stochastic noise on the differential confusability of
similar stimuli. Our main contribution, therefore, is a technical
answer to Lipman's problem in the sense introduced above. The dynamic
proposed here nicely improves on and complements previous like-minded
accounts. We show that that stochastic noise can not only lead to
vague, yet communicative efficient signal use, but can also unify
evolutionary outcomes and help avoid inefficient categorization.

The next section introduces the background against which the work
presented here can be appreciated. Section~\ref{sec:repl-diff-dynam},
then, introduces the replicator diffusion dynamic, and elaborates its
relation with the replicator mutator
dynamic. Section~\ref{sec:exploring-rdd} explores the replicator
diffusion dynamic on the relevant class of generalized signaling games
introduced in
Section~\ref{sec:background}. Section~\ref{sec:discussion} reflects on
the results and compares what has been achieved to related accounts in
more detail. 

\section{Background}
\label{sec:background}

The view that vagueness is a natural concomitant of cognitive
limitations of language users has been formalized in a number of ways,
using evolutionary game theory and certain generalizations of
signaling games, so called similarity-maximizing games or sim-max
games, for short
\citep{FrankeJager2010:Vagueness-Signa,OConnor2013:Evolving-Percep,OConnor2013:The-Evolution-o}. Our
contribution is best seen in relation to these accounts, as it also
relies on sim-max games. Let's introduce these first.

Signaling games, as introduced by \citet{Lewis_1969:Convention}, have
a sender and a receiver. The sender knows the true state of the world,
but the receiver does not. The sender can select a signal, or message,
to reveal to the receiver, who then chooses an act. In Lewis' games,
states are maximally distinct from each other and the receiver's acts
are related to them one-to-one. If the receiver chooses the act that
corresponds to the actual state, the play is a success, otherwise a
failure. Similarity-maximizing games are generalizations of Lewis'
games in that they allow different states to be more or less similar
to one another, and, roughly put, to make success of communication a
function of that similarity. Signaling games with utility-relevant
similarities in the state space are fairly standard in economics
\citep[e.g.][]{Spence1973:Job-market-sign,CrawfordSobel1982:Strategic-Infor},
but have received particular attention in a more philosophical and
linguistic context by the more recent work of
\citet{Jager2007:The-Evolution-o,JagerRooijvan-Rooij2007:Language-Struct,JagerMetzger2011:Voronoi-Languag}. 

More concretely, a sim-max game, in the relevant sense here, consists
of a set of states $\States$, a set of messages $\Messgs$, a
similarity metric on states: $\similarity \mycolon \States \times
\States \rightarrow \mathds{R}$, and a utility function $\Utils
\mycolon \States \times \States \rightarrow \mathds{R}$. We identify
the receiver's acts with the states of the world, so that the game is
one of guessing the actual state, so to speak. We also assume that
sender and receivers interests are alike, so we only have one utility
function. We do not consider message costs, so utilities only depend
on the actual state, and the receiver's response. The similarity function
should satisfies the usual requirements for a metric:
\begin{align*}
  & \similarity(\mystate{1}, \mystate{2}) \ge 0 &&
  \similarity(\mystate{1}, \mystate{2}) = 0 \Leftrightarrow
  \mystate{1} = \mystate{2} \\
  & \similarity(\mystate{1}, \mystate{2}) = \similarity(\mystate{2},
  \mystate{1}) && \similarity(\mystate{1}, \mystate{2}) \le
  \similarity(\mystate{1}, \mystate{3}) + \similarity(\mystate{3}, \mystate{2})\,.
\end{align*}
The utility function should be a monotonically decreasing function of
similarity:
\begin{align*}
  \similarity(\mystate{1},\mystate{2}) \ge
  \similarity(\mystate{1},\mystate{3}) \ \ \Rightarrow \ \ 
  \Utils(\mystate{1},\mystate{2}) \ge
  \Utils(\mystate{1},\mystate{3})\,.
\end{align*}
To keep matters simple, we only look at cases where $\States$ contains
finitely many points from the unit interval, with similarity given by
the Euclidean distance in this one-dimensional space.\todo{introduce
  utilities here?}

\begin{itemize}
\item ESSs of sim-max games
\item Voronoi tessallations
\item conceptual spaces
\item 
\end{itemize}

\bigskip


As in other applications of
evolutionary game theory, we can distinguish micro- and macro-level
approaches. Micro-level approaches look at adaptive behavior of
individual agents. Usually, changes in the behavioral dispositions of
agents occur after every single interaction. In contrast, macro-level
approaches outline more abstract, aggregate dynamics, happening in a
huge population of agents, or otherwise abstracting from seemingly
irrelevant detail. Usually, a macro-level dynamic captures changes of
frequencies of behavioral types in the population over time. Previous
accounts have suggested both micro- and macro-level dynamics to show
how vague signal use can arise in sim-max games.

As for micro-dynamics, \citet{FrankeJager2010:Vagueness-Signa} show
how limited memory of past interactions can lead to vague signal use,
when averaging over a single agent's behavior over time or over the
momentary behavior of a population of several language
users. \citet{OConnor2013:The-Evolution-o} introduces a variant of
reinforcement learning that includes a low-level form of stimulus
generalization. Agents update their behavior after each round of play
in such a way that similar states to the ones that actually occurred
are subject to behavioral adjustment as well. (Alternative approaches
are discussed in more detail in Section~XYZ, where we can compare them
with the system introduced in this paper.)

\citet{FrankeJager2010:Vagueness-Signa} also consider a macro-level
approach, using the notion of a \emph{quantal response}. A quantal
response function is a probabilistic choice rule that formalizes the
idea that agents make small mistakes when calculating the expected
utility of choice options. In aggregation, these probabilistic
mistakes lead to systematic ``trembles'' that produce vague signal
use. The approach we take here is superficially similar, but there are
crucial divergences. For one, we adopt a dynamic perspective by
looking at the stable states of a suitable population dynamics. For
another, we demonstrate in Section~XYZ that quantal responses can give
rise to counterintuitive predictions. These counterintuitive examples
suggest that vagueness in signaling use is not convincingly explained
by appeal to mistakes in calculating expected utility, but rather, as
we assume here, as the result of confusing similar states of the word.

Against this background, we introduce a new macro-level approach
to the evolution of vague signal use in sim-max games, where the
source of vagueness is the agents' natural disability to sharply
distinguish perceptually distinct states. This applies to both
production and comprehension, i.e., both to the sender as well as to
the receiver role. 

We call our dynamic \textbf{replicator diffusion dynamic} (\rdd),
because it is a special case of the replicator mutator dynamic
(\rmd). It is a special case in the sense that the \rdd incorporates a
special kind of mutation that is based on the similarity of states (as
defined in the sim-max game). More concretely, we start from the idea
that similar states are more easily confused than less similar
states. To formalize this idea, we assume that there is a confusion
matrix $C$, where $C_{ij}$ is the probability that a state \mystate{i}
is realized as a possibly different state \mystate{j}. (There is a
slightly different sense of ``realization'' in the sender and the
receiver side, and we will enlarge on this below.) Based on this, we
show two things. For one, we show how the discrete-time formulation of
the replicator dynamic can be easily adapted to integrate diffusion of
behavior due to confusion of states. This yields the \rdd in its basic
discrete-time formulation. For another, we also show that there exists
a translation of the confusion matrix $C$ into a suitable mutation
matrix $M_C$ such that the effect of $C$ and $M_C$ are ``equivalent''
in a sense to be made precise. This supports the conclusion that the
\rdd can indeed be thought of as a special case of the \rmd.



%%% Local Variables: 
%%% mode: latex
%%% TeX-master: "paper"
%%% TeX-PDF-mode: t
%%% End:



