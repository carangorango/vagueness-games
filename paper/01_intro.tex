\section{Introduction}

\begin{itemize}
\item introduce sim-max games
\item problem of vagueness \citep{Lipman2009:Why-is-Language}
\item \dots
\end{itemize}

\subsection{Situating our approach}

A number of authors have recently tried to explain why vagueness
evolved, based on considerations why a vague language might be useful
\citep[e.g.][]{Jaegherde-Jaegher2003:A-Game-Theoreti,Deemter2009:Utility-and-Lan,Jaegherde-JaegherRooijvan-Rooij2010:Strategic-Vague,BlumeBoard2013:Intentional-Vag}. In
contrast to these, others have argued that vagueness if a natural
byproduct of limitations in information processing
\citep[e.g.][]{FrankeJager2010:Vagueness-Signa,OConnor2013:The-Evolution-o}. This
paper makes a contribution to the latter.

The view that vagueness is a natural concomitant of cognitive
limitations of language users has been formalized in a number of
ways. As in other applications of evolutionary game theory, we can
distinguish micro- and macro-level approaches. Micro-level approaches
look at adaptive behavior of individual agents. Usually, changes in
the behavioral dispositions of agents occur after every single
interaction. In contrast, macro-level approaches outline more
abstract, aggregate population-based dynamics. Usually, the dynamic
captures changes of frequencies of behavioral types in the population
over time. Previous accounts have suggested both micro- and
macro-level dynamics to show how vague signal use can arise in sim-max
games.\todo{what about Ewan Klein's stuff?}

As for micro-dynamics, \citet{FrankeJager2010:Vagueness-Signa} show
how limited memory of past interactions can lead to vague signal use,
when averaging over a single agent's behavior over time or over the
momentary behavior of a population of several language
users. \citet{OConnor2013:The-Evolution-o} introduces a variant of
reinforcement learning that includes a low-level form of stimulus
generalization. Agents update their behavior after each round of play
in such a way that similar states to the ones that actually occurred
are subject to behavioral adjustment as well. (Alternative approaches
are discussed in more detail in Section~XYZ, where we can compare them
with the system introduced in this paper.)

\citet{FrankeJager2010:Vagueness-Signa} also consider a macro-level
approach, using the notion of a \emph{quantal response}. A quantal
response function is a probabilistic choice rule that formalizes the
idea that agents make small mistakes when calculating the expected
utility of choice options. In aggregation, these probabilistic
mistakes lead to systematic ``trembles'' that produce vague signal
use. The approach we take here is only superficially similar. For one,
we not only look at equilibria, but take a dynamic perspective
here. For another, we demonstrate in Section~XYZ that quantal
responses can give rise to counterintuitive predictions in sim-max
games, which suggests that the source of stochastic trembles should
better not be seen in mistakes in calculating expected utility, but
rather, as we assume here, as confusion of similar states.

Against this background, we introduce here a new macro-level approach
to the evolution of vague signal use in sim-max games, where the
source of vagueness is the agents' natural disability to sharply
distinguish perceptually distinct states. This applies to both
production and comprehension, i.e., both to the sender as well as to
the receiver role. 

We call our dynamic \textbf{replicator diffusion dynamic} (\rdd),
because it is a special case of the replicator mutator dynamic
(\rmd). It is a special case in the sense that the \rdd uses a special
kind of mutation matrix that is based on the similarity of states (as
defined in the sim-max game). More concretely, we start from the idea
that similar states are more easily confused than less similar
states. To formalize this idea, we assume that there is a confusion
matrix $C$, where $C_{ij}$ is the probability that a state \mystate{i}
is realized as a possibly different state \mystate{j}. (There is a
slightly different sense of ``realization'' in the sender and the
receiver side, and we will enlarge on this below.) Based on this, we
show two things. For one, we show how the discrete-time formulation of
the replicator dynamic can be easily adapted to integrate diffusion of
behavior due to confusion of states. This yields the \rdd in its basic
discrete-time formulation. For another, we also show that there exists
a translation of the confusion matrix $C$ into a suitable mutation
matrix $M_C$ such that the effect of $C$ and $M_C$ are ``equivalent''
in a sense to be made precise. This supports the conclusion that the
\rdd can indeed be thought of as a special case of the \rmd.

To understand the workings of the \rdd, we study numerical simulations
\dots

The paper is structured as follows:
\begin{enumerate}
\item replicator diffusion dynamics
\item relation to replicator mutator dynamics
\item simulations of \rdd
\item comparison with other approaches
\item conclusions
\end{enumerate}


%%% Local Variables: 
%%% mode: latex
%%% TeX-master: "paper"
%%% TeX-PDF-mode: t
%%% End:



