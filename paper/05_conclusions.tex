\section{Conclusions}
\label{sec:conclusions}

It is a special case in the sense that the \rdd incorporates a
particular kind of mutation that is based on the similarity of states
(as defined in the sim-max game). More concretely, we start from the
idea that similar states are more easily confused than less similar
states. To formalize this idea, we assume that there is a confusion
matrix $C$, where $C_{ij}$ is the probability that a state \mystate{i}
is realized as a possibly different state \mystate{j}. (There is a
slightly different sense of ``realization'' in the sender and the
receiver side, and we will enlarge on this below.) Based on this, we
show two things. For one, we show how the discrete-time formulation of
the replicator dynamic can be easily adapted to integrate diffusion of
behavior due to confusion of states. This yields the \rdd in its basic
discrete-time formulation. For another, we also show that there exists
a translation of the confusion matrix $C$ into a suitable mutation
matrix $M_C$ such that the effect of $C$ and $M_C$ are ``equivalent''
in a sense to be made precise. This supports the conclusion that the
\rdd can indeed be thought of as a special case of the \rmd.

Based on data from numerical simulations, we show that the inclusion
of mild levels of such stochastic noise in the replicator dynamics
does not disrupt the possibility of evolving communicative efficient
signaling strategies at all. On the contrary, there might even be a
higher-order benefit to the presence of perceptual imprecision, in
that it accelerates converges, but also unifies and regularizes
evolutionary outcomes in such a way that inefficient categorization is
avoided. These results complement similar results by
\citet{OConnor2013:The-Evolution-o}, obtained for a micro-dynamic
extension of reinforcement learning. The \rdd adds to this a more
general, abstract framework that is not tied to the specific
assumptions of turn-based reinforcement. 

%%% Local Variables: 
%%% mode: latex
%%% TeX-master: "paper"
%%% TeX-PDF-mode: t
%%% End:



