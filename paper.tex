%% LyX 2.0.3 created this file.  For more info, see http://www.lyx.org/.
%% Do not edit unless you really know what you are doing.
\documentclass[a4paper,english]{article}
\usepackage[T1]{fontenc}
\usepackage[latin9]{inputenc}
\usepackage{units}
\usepackage{amsmath}

\makeatletter

%%%%%%%%%%%%%%%%%%%%%%%%%%%%%% LyX specific LaTeX commands.
\pdfpageheight\paperheight
\pdfpagewidth\paperwidth


%%%%%%%%%%%%%%%%%%%%%%%%%%%%%% Textclass specific LaTeX commands.
\usepackage[color=green!40]{todonotes}

%%%%%%%%%%%%%%%%%%%%%%%%%%%%%% User specified LaTeX commands.
\usepackage{a4wide}

\makeatother

\usepackage{babel}
\begin{document}

\title{The evolution of vague categories}


\author{Jos� Pedro Correia\and Michael Franke}

\maketitle

\section{Quantitative measures of uncertainty}


\subsection{Distance-based}

Both sender and receiver strategies are of the form $s:C\rightarrow\Delta\left(A\right)$,
\emph{i.e.}\ a function $s$ from a set of choice points $C$ to
a probability distribution over a set of actions $A$. We will henceforth
restrict ourselves to finite $C$ and $A$. For a given choice point
$c\in C$, a strategy $\hat{s}$\todo{Find better notation?} is maximally
uncertain if it assigns the same probability to every action $a\in A$,
\emph{i.e.}\ if $\hat{s}\left(c\right)$ is a discrete uniform distribution:
\[
\forall a\in A:\hat{s}\left(c,a\right)=\frac{1}{\left|A\right|}
\]
We say it is maximally uncertain since it has no preference whatsoever
for any action in $A$.

We can measure how certain a strategy $s$ is at a choice point $c$
in terms of the distance from such a distribution $\hat{s}\left(c\right)$,
for example in terms of the Kullback-Leibler divergence:
\begin{gather*}
D_{\textnormal{KL}}\left(s\left(c\right)\parallel\hat{s}\left(c\right)\right)=\sum_{a\in A}s\left(c,a\right)\cdot\ln\left(\frac{s\left(c,a\right)}{\hat{s}\left(c,a\right)}\right)=\\
=\sum_{a\in A}s\left(c,a\right)\cdot\ln\left(\frac{s\left(c,a\right)}{\frac{1}{\left|A\right|}}\right)=\sum_{a\in A}s\left(c,a\right)\cdot\ln\left(\left|A\right|\cdot s\left(c,a\right)\right)
\end{gather*}
This distance should be maximal\todo{Is it the case?} for any strategy
$\check{s}$\todo{Find better notation?} which puts all probability
mass on some action $a^{\prime}\in A$, \emph{i.e.}\ a degenerate
distribution localized at $a^{\prime}$:
\[
\check{s}\left(c,a\right)=\begin{cases}
1 & a=a^{\prime}\\
0 & \textnormal{otherwise}
\end{cases}
\]
For such a strategy we have:
\begin{gather*}
D_{\textnormal{KL}}\left(\check{s}\left(c\right)\parallel\hat{s}\left(c\right)\right)=\sum_{a\in A}\check{s}\left(c,a\right)\cdot\ln\left(\left|A\right|\cdot\check{s}\left(c,a\right)\right)\\
=\sum_{a\in A\backslash\left\{ a^{\prime}\right\} }\check{s}\left(c,a\right)\cdot\ln\left(\left|A\right|\cdot\check{s}\left(c,a\right)\right)+\check{s}\left(c,a^{\prime}\right)\cdot\ln\left(\left|A\right|\cdot\check{s}\left(c,a^{\prime}\right)\right)=\\
=\sum_{a\in A\backslash\left\{ a^{\prime}\right\} }0\cdot\ln\left(\left|A\right|\cdot0\right)+1\cdot\ln\left(\left|A\right|\cdot1\right)=\ln\left|A\right|
\end{gather*}
This value allows us to normalize the distance so that our measure
of uncertainty ranges between $0$ and $1$. Finally, the measure
of uncertainty of a strategy $s$ at a choice point $c$ thus becomes:
\[
u\left(s,c\right)=1-\frac{1}{\ln\left|A\right|}\sum_{a\in A}s\left(c,a\right)\cdot\ln\left(\left|A\right|\cdot s\left(c,a\right)\right)
\]


We can then measure the overall uncertainty of a strategy $s$ as
the mean uncertainty per choice point:
\[
u\left(s\right)=\frac{1}{\left|C\right|}\sum_{c\in C}u\left(s,c\right)
\]



\subsection{Entropy-based}

Another option is to define uncertainty in terms of entropy. The most
natural way is to define it in terms of mixed strategies rather than
behavioral strategies. Thus, let $\vec{s}:\Delta\left(A^{C}\right)$
be a mixed strategy represented as a probability distribution over
all possible pure strategies from choice points to actions $s:A^{C}$.
We can define the entropy of such a strategy as:
\[
E\left(\vec{s}\right)=\sum_{s\in A^{C}}\vec{s}\left(s\right)\cdot\ln\left(\vec{s}\left(s\right)\right)
\]
This is computationally expensive to compute, since the size of the
domain over which the sum is computed grows exponentially with the
number of choice points. Therefore, we calculate an equivalent measure
defined in terms of behavioral strategies. In general, a behavioral
strategy can be converted into an infinite number of mixed strategies.
However, we can define a unique mapping from a behavioral strategy
$\sigma:C\rightarrow\Delta\left(A\right)$ to a mixed strategy $\vec{s}_{\sigma}$
where $\vec{s}_{\sigma}\left(s\right)=\prod_{c\in C}\sigma\left(s\left(c\right)|c\right)$,
for every $s\in A^{C}$. Based on that, we have the following equivalences:
\begin{eqnarray}
E\left(\sigma\right)= & E\left(\vec{s}_{\sigma}\right)\nonumber \\
= & -\sum_{s\in A^{C}}\vec{s}_{\sigma}\left(s\right)\cdot\ln\left(\vec{s}_{\sigma}\left(s\right)\right)\nonumber \\
= & -\sum_{s\in A^{C}}\left(\prod_{c\in C}\sigma\left(s\left(c\right)|c\right)\right)\cdot\ln\left(\prod_{c\in C}\sigma\left(s\left(c\right)|c\right)\right)\nonumber \\
= & -\sum_{s\in A^{C}}\left(\prod_{c\in C}\sigma\left(s\left(c\right)|c\right)\right)\cdot\sum_{c\in C}\ln\left(\sigma\left(s\left(c\right)|c\right)\right)\nonumber \\
= & -\sum_{s\in A^{C}}\sum_{c\in C}\ln\left(\sigma\left(s\left(c\right)|c\right)\right)\cdot\prod_{c^{\prime}\in C}\sigma\left(s\left(c^{\prime}\right)|c^{\prime}\right)\nonumber \\
= & -\sum_{c\in C}\sum_{s\in A^{C}}\ln\left(\sigma\left(s\left(c\right)|c\right)\right)\cdot\sigma\left(s\left(c\right)|c\right)\cdot\prod_{c^{\prime}\in C\backslash\left\{ c\right\} }\sigma\left(s\left(c^{\prime}\right)|c^{\prime}\right)\label{eq:step-x}\\
= & -\sum_{c\in C}\sum_{a\in A}\ln\left(\sigma\left(a|c\right)\right)\cdot\sigma\left(a|c\right)\cdot\sum_{s\in A^{\left\{ c\mapsto a\right\} \cup C\backslash\left\{ c\right\} }}\prod_{c^{\prime}\in C\backslash\left\{ c\right\} }\sigma\left(s\left(c^{\prime}\right)|c^{\prime}\right)\nonumber \\
= & -\sum_{c\in C}\sum_{a\in A}\ln\left(\sigma\left(a|c\right)\right)\cdot\sigma\left(a|c\right)\cdot\prod_{c^{\prime}\in C\backslash\left\{ c\right\} }\sum_{a\in A}\sigma\left(a|c^{\prime}\right)\nonumber \\
= & -\sum_{c\in C}\sum_{a\in A}\ln\left(\sigma\left(a|c\right)\right)\cdot\sigma\left(a|c\right)\cdot\prod_{c^{\prime}\in C\backslash\left\{ c\right\} }1\nonumber \\
= & -\sum_{c\in C}\sum_{a\in A}\ln\left(\sigma\left(a|c\right)\right)\cdot\sigma\left(a|c\right)\label{eq:step-y}
\end{eqnarray}
To illustrate the transitions from (\ref{eq:step-x}) to (\ref{eq:step-y}),
let us look at an example. Let $C=\left\{ c_{1},c_{2},c_{3}\right\} $,
$A=\left\{ a_{1},a_{2}\right\} $, $\sigma_{xy}=\sigma\left(a_{y}|c_{x}\right)$,
and $s_{xyz}=\left\{ c_{1}\mapsto a_{x},c_{2}\mapsto a_{y},c_{3}\mapsto a_{z}\right\} $.
Take a given $c\in C$, for example $c=c_{1}$. The inner sum in (\ref{eq:step-x})
expands to:
\begin{eqnarray*}
 & \ln\left(\sigma\left(s_{111}\left(c_{1}\right)|c_{1}\right)\right)\cdot\sigma\left(s_{111}\left(c_{1}\right)|c_{1}\right)\cdot\sigma\left(s_{111}\left(c_{2}\right)|c_{2}\right)\cdot\sigma\left(s_{111}\left(c_{3}\right)|c_{3}\right) & +\\
+ & \ln\left(\sigma\left(s_{112}\left(c_{1}\right)|c_{1}\right)\right)\cdot\sigma\left(s_{112}\left(c_{1}\right)|c_{1}\right)\cdot\sigma\left(s_{112}\left(c_{2}\right)|c_{2}\right)\cdot\sigma\left(s_{112}\left(c_{3}\right)|c_{3}\right) & +\\
+ & \vdots & +\\
+ & \ln\left(\sigma\left(s_{221}\left(c_{1}\right)|c_{1}\right)\right)\cdot\sigma\left(s_{221}\left(c_{1}\right)|c_{1}\right)\cdot\sigma\left(s_{221}\left(c_{2}\right)|c_{2}\right)\cdot\sigma\left(s_{221}\left(c_{3}\right)|c_{3}\right) & +\\
+ & \ln\left(\sigma\left(s_{222}\left(c_{1}\right)|c_{1}\right)\right)\cdot\sigma\left(s_{222}\left(c_{1}\right)|c_{1}\right)\cdot\sigma\left(s_{222}\left(c_{2}\right)|c_{2}\right)\cdot\sigma\left(s_{222}\left(c_{3}\right)|c_{3}\right)
\end{eqnarray*}
or:
\begin{eqnarray*}
 & \ln\left(\sigma_{11}\right)\cdot\sigma_{11}\cdot\sigma_{21}\cdot\sigma_{31} & +\\
+ & \ln\left(\sigma_{11}\right)\cdot\sigma_{11}\cdot\sigma_{21}\cdot\sigma_{32} & +\\
+ & \ln\left(\sigma_{11}\right)\cdot\sigma_{11}\cdot\sigma_{22}\cdot\sigma_{31} & +\\
+ & \ln\left(\sigma_{11}\right)\cdot\sigma_{11}\cdot\sigma_{22}\cdot\sigma_{32} & +\\
+ & \ln\left(\sigma_{12}\right)\cdot\sigma_{12}\cdot\sigma_{21}\cdot\sigma_{31} & +\\
+ & \ln\left(\sigma_{12}\right)\cdot\sigma_{12}\cdot\sigma_{21}\cdot\sigma_{32} & +\\
+ & \ln\left(\sigma_{12}\right)\cdot\sigma_{12}\cdot\sigma_{22}\cdot\sigma_{31} & +\\
+ & \ln\left(\sigma_{12}\right)\cdot\sigma_{12}\cdot\sigma_{22}\cdot\sigma_{32}
\end{eqnarray*}
We can split this sum into one part per action, namely:
\begin{eqnarray*}
 & \ln\left(\sigma_{11}\right)\cdot\sigma_{11}\cdot\left(\sigma_{21}\cdot\sigma_{31}+\sigma_{21}\cdot\sigma_{32}+\sigma_{22}\cdot\sigma_{31}+\sigma_{22}\cdot\sigma_{32}\right) & +\\
+ & \ln\left(\sigma_{12}\right)\cdot\sigma_{12}\cdot\left(\sigma_{21}\cdot\sigma_{31}+\sigma_{21}\cdot\sigma_{32}+\sigma_{22}\cdot\sigma_{31}+\sigma_{22}\cdot\sigma_{32}\right)
\end{eqnarray*}
The sums in parenthesis can be further rewritten as:
\begin{eqnarray*}
 & \ln\left(\sigma_{11}\right)\cdot\sigma_{11}\cdot\left(\sigma_{21}+\sigma_{22}\right)\cdot\left(\sigma_{31}+\sigma_{32}\right) & +\\
+ & \ln\left(\sigma_{12}\right)\cdot\sigma_{12}\cdot\left(\sigma_{21}+\sigma_{22}\right)\cdot\left(\sigma_{31}+\sigma_{32}\right)
\end{eqnarray*}
Given that $\sigma$ is row stochastic, each of the terms is equal
to $1$, thus reducing the whole expression to:
\[
\ln\left(\sigma_{11}\right)\cdot\sigma_{11}+\ln\left(\sigma_{12}\right)\cdot\sigma_{12}
\]


The measure of entropy is lower bounded at $0$ and upper bounded
at $\ln\left(\left|A^{C}\right|\right)=\ln\left(\left|A\right|^{\left|C\right|}\right)=\left|C\right|\cdot\ln\left(\left|A\right|\right)$,
which means we can normalize $E\left(\sigma\right)$ between $0$
and $1$ by dividing by $\nicefrac{1}{\left|C\right|\cdot\ln\left(\left|A\right|\right)}$.
\end{document}
