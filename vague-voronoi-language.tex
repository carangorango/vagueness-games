\documentclass{article}
\usepackage[utf8]{inputenc}
\usepackage{a4wide}
\usepackage{amsmath}
\usepackage{todonotes}

\DeclareMathOperator*{\argmax}{arg\,max}
\DeclareMathOperator*{\argmin}{arg\,min}

\title{Vagueness games}
\author{José Pedro Correia \and Michael Franke}
\date{\today}

\begin{document}

\maketitle

\section{Vague Voronoi languages}

A \emph{vague Voronoi language} consists of a sender strategy $\sigma:T\rightarrow\Delta\left(M\right)$ and a receiver strategy $\rho:M\rightarrow \Delta\left(T\right)$ such that:

\begin{enumerate}
\item The receiver strategy assigns maximal probability to one state per message. The maximal element of a strategy for a message $m\in M$ can be understood as the prototype of that message:
\begin{equation}
\forall m\in M:\exists!t\in T:\forall t^{\prime}\in T:\rho\left(m,t\right)\geq\rho\left(m,t^{\prime}\right)
\end{equation}
Note that after we guarantee the uniqueness of message prototypes for a strategy $\rho$, we can define a prototype function $P_{\rho}:M\rightarrow T$
which associates to each message its prototype as $P_{\rho}\left(m\right)=\argmax_{t\in T}\rho\left(m,t\right)$

\item In the receiver strategy, for each message, the probability of a state being picked for that message increases with the state's similarity with the message's prototype:
\begin{equation}
\forall m\in M:\forall t,t^{\prime}\in T:s\left(t,P_{\rho}\left(m\right)\right)>s\left(t^{\prime},P_{\rho}\left(m\right)\right)\Rightarrow\rho\left(m,t\right)>\rho\left(m,t^{\prime}\right)
\end{equation}
\todo[inline]{Is this so far the same as defining $\rho\left(m\right)$ as a convez fuzzy set for every $m\in M$? Probably not...}

\item In the sender strategy, the more similar a state is to a message's prototype in comparison with other prototypes, the higher the probability that message will be used:
\begin{equation}
\forall t\in T:\forall m,m^{\prime}\in M:s\left(t,P_{\rho}\left(m\right)\right)>s\left(t,P_{\rho}\left(m^{\prime}\right)\right)\Rightarrow\sigma\left(t,m\right)>\sigma\left(t,m^{\prime}\right)
\end{equation}
\todo[inline]{A non-vague language Voronoi language should be a special case of this. We should show that and add a criterion of non-triviality, right?}
\end{enumerate}

\end{document}
